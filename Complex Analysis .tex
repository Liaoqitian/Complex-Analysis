\documentclass{article}
\usepackage{amsmath} 
\usepackage{graphicx}
\usepackage{subcaption}
\usepackage{sectsty}
\usepackage{amssymb}
 \usepackage{lipsum}
\usepackage{titlesec}
\usepackage{romannum}
\usepackage{enumitem}

\setlist[itemize,1]{leftmargin=\dimexpr 26pt-.5in}


\sectionfont{\fontsize{12}{15}\selectfont}
\title{Introduction to Complex Analysis}
\author{Qitian Liao}
\date{Aug 3, 2020} 
\usepackage[left=2cm, right=2cm, top=2cm]{geometry}
\setlength\parindent{0pt}
\begin{document}
%\pagenumbering{gobble}
\maketitle
\newpage
\tableofcontents
\def\Arg{\mathop{\operator@font Arg}\nolimits}
%\newpage
\pagenumbering{arabic}
\titleformat*{\section}{\Large\bfseries}
\titleformat*{\subsection}{\large\bfseries}
\titleformat*{\subsubsection}{\normalsize\bfseries}
\titleformat*{\paragraph}{\large\bfseries}
\titleformat*{\subparagraph}{\large\bfseries}

\titlespacing\section{0pt}{5pt plus 4pt minus 2pt}{5pt plus 2pt minus 2pt}
\titlespacing\subsection{0pt}{10pt plus 4pt minus 2pt}{5pt plus 2pt minus 2pt}
\titlespacing\subsubsection{0pt}{5pt plus 4pt minus 2pt}{5pt plus 2pt minus 2pt}

\newpage
\section{Algebra of the Complex Plane}
\subsection{Introduction to Complex Numbers}
Let $z = a + ib \in \mathbb{C}$ where $a, b \in \mathbb{R}$ and $i ^ 2 = -1$. \\
This number can be thought of as a point in 2-space, $\mathbb{R} ^ 2$, $(a, b)$ or as a position in $\mathbb{C}$. \\ 
$\mathbb{R} ^ 2$: $\oplus$ addition; $\odot$ scalar multiplication. \\
$\mathbb{C}$ : $\oplus$ addition; $\odot$ scalar multiplication; a vector space; have multiplication of elements, $\mathbb{C}$ is a field. \\
\begin{equation*} 
\mbox{If } z = a + ib \mbox{, } w = c + id \mbox{, then }zw = (ac - bd) + i(ad + cb)
\end{equation*}
$$zw = wz$$
$$z(w + \alpha) = zw + z\alpha$$ 
$$(zw)\alpha = z(w\alpha)$$ 

\subsection{Conjugate of Complex Numbers} 
\newcommand*\conj[1]{\overline{#1}}
\subsubsection{Definition of Conjugate} 
The complex conjugate of $z$, $\conj{z}$, is defined by 
$$\conj{z} = a - ib$$ 
Geometric representation: The image of $\bar z$ is the reflection of z about the Real axis.
\subsubsection{Properties of Conjugate} 
$$\conj{\conj{z}} = z$$
$$\conj{zw} = \conj{z}\conj{w}$$
$$\conj{z + w} = \conj{z} + \conj{w}$$
\begin{equation*}
\overline{z} = z \mbox{ if and only if } z \in \mathbb{R}
\end{equation*}
\subsubsection{Real and Imaginary Parts}
\noindent We can project $z$ onto the Real or Imaginary axis and measure its distance from 0: 
$$\Re(z) = a$$
\begin{equation*}
\Im(z) = b \mbox{, not } ib
\end{equation*}

Each function is a map $\mathbb{C} \to \mathbb{R}$. Then 
$$\Re(z) = \frac{z + \conj{z}}{2}$$
$$\Im(z) = \frac{z - \conj{z}}{2i}$$
This is similar to the pattern with even/odd functions. 

\subsection{Modulus of Complex Numbers}
Note: $z\conj{z} = (a + ib)(a - ib) = a^2 + b^2 \in \mathbb{R}$ 

\subsubsection{Definition of Modulus}
$|z|$ length/modulus of $z$ is defined by:
$$|z| = (a^2 + b^2)^{\frac{1}{2}} = (z\conj{z})^{\frac{1}{2}} \in \mathbb{R}$$

\subsubsection{Properties of Modulus} 
$$|zw| = |z||w|$$ 
$$|z| = |\conj{z}|$$  
$$|z| \geqslant 0$$
\begin{equation*}
|z| = 0 \mbox{ if and only if } z = 0
\end{equation*}

\subsubsection{Triangle Inequality}
Triangle Inequality: 
$$|z + w| \leqslant |z| + |w|$$ 
 
Reverse Triangle Inequality: 
$$|z| - |w| \leqslant |z - w|$$
$z = z - w + w \Rightarrow |z| = |z - w + w| \Rightarrow |z| \leqslant |z - w| + |w| \Rightarrow |z| - |w| \leqslant |z - w|$. 

\subsubsection{Complex Division} 
With $z\conj{z} \in \mathbb{R}$, we can define complex division by reducing it to a multiplication problem. 
$$\frac{z}{w} = \frac{z\conj{w}}{w\conj{w}} = \frac{1}{w\conj{w}}(z\conj{w})$$ 
We also have 
\begin{equation*} 
|\frac{z}{w}| = \frac{|z|}{|w|} \mbox{ for } w \neq 0 
\end{equation*}

\subsubsection{Distance in the plane} 
A disk in the complex plane centered at $c$ of radius $r \in \mathbb{R}$ is of the form 
$$\{z \in \mathbb{C} \mid |z - c| \leqslant r\}$$

\subsection{Complex Polynomial}
A complex polynomial $p(z)$ of degree n is of the form: 
$$p(z) = a_nz^n + a_{n - 1}z^{n - 1} + \cdots + a_1z + a_0$$
where $a_n \neq 0$ and $a_i \in \mathbb{C}$ for $ i = 0, \cdots , n$

\subsubsection{Fundamental Theorem of Algebra} 

The factorization of $p(z)$ factors over $\mathbb{C}$ is unique, 
$$p(z) = c(z - z_1)^{m_1}...(z - z_k)^{m_k}$$
We have roots $z_i \in \mathbb{C}$ of $p(z)$ with order $m_i \in \mathbb{N}$. \\
For example, if $p(z) = z^2 + 4 = (z + 2i)(z - 2i)$, then it factors over $\mathbb{C}$ but not $\mathbb{R}$. \\
\textbf{Note}: 
$\mathbb{C}$ is an algebraically closed field, there are no irreducible polynomials in $\mathbb{C}$. \\
\textbf{Note}: 
$\mathbb{R}$,  $\mathbb{Q}$, $\mathbb{Z}$, $\mathbb{N}$ are not algebraically closed. \\

\newpage
\section{Geometry of the Complex Plane}
\subsection{Properties of Polar Forms} 
Complex numbers can be represented in polar forms: 
$$z = |z|(\cos\theta + i\sin\theta)$$
with modulus $|z|$ and argument $\theta$. To change between the coordinate systems it follows: 
$$|z| = (a^2 + b^2)^{\frac{1}{2}}$$
$$\tan\theta = \frac{b}{a}$$
$$a = |z|\cos\theta= \Re(z)$$
$$b = |z|\sin\theta = \Im(z)$$
\textbf{Note}: $\theta_R = \arctan(\frac{b}{a})$ is a reference angle of $z$. To find $\theta$ from $\theta_R$, you need to consider the signs of $a$ and $b$. \\
Example: \\
$z = -3 + 3i = 3\sqrt2(\cos\frac{3\pi}{4} + \sin\frac{3\pi}{4})$ \\
$\theta_R = \arctan(\frac{3}{-3})= -\frac{\pi}{4}$ \\
$\theta = \pi + \theta_R = \pi -\frac{\pi}{4} = \frac{3\pi}{4}$, since $\theta$ is in $\Romannum{2}$.

\subsection{Definition of Argument and argument}
$\operatorname{Arg}(z)$ is $z$'s principle polar angle $\theta$, $z \neq 0$, where $\theta \in (-\pi, \pi]$. \\
$\operatorname{arg}(z)$ is all of $z$'s polar angles, $\theta + 2k\pi$, $k \in \mathbb{Z}$. 

\subsection{Euler's Formula} 
Euler's Formula is defined as a linear combination of $\cos\theta$ and $\sin\theta$, $\mathbb{R}$-valued functions. 
$$ e^{i\theta}= \cos\theta + i\sin\theta $$
It allows us to express $z$ in polar form by 
$$ z = |z|e^{i\theta}$$
-1 has polar angle $\pi$ and modulus 1, 
\begin{equation*}
-1 = e^{i\pi} \mbox{ or } e^{i\pi} + 1 = 0
\end{equation*}
By the angle addition formulas from trigonometry we find: 
$$e^{i\theta}e^{i\varphi} = e^{i(\theta + \varphi)}$$
$$(e^{i\theta})^k = e^{i\theta k}$$

\subsection{Geometric Understanding of Multiplication}
The polar angle of $zw$ is the sum of the polar angles of $z$ and $w$. The modulus is the product of the moduli. 
$$ \operatorname{Arg}(zw) = \operatorname{Arg}(z) + \operatorname{Arg}(w)$$
$$ \operatorname{Arg}(\conj{z}) = -\operatorname{Arg}(z)$$
Question: How about $\frac{z}{w}$ and $z^4$? \\
It follows from trigonometry that $|e^{i\theta}| = 1$, if $\theta \in (-\pi, \pi]$ we get a parametrization of the unit circle. \\
\newline
\textbf{Example}: Discover all solutions to $w^3 = i = z$ \\
Let $p(z) = w^3 - i$. By Fundamental Theorem of Algebra, there are 3 roots of $p(z)$. \\
Therefore, $3\theta= \frac{\pi}{2} + 2\pi k$, $k \in \mathbb{Z}$ \\
This gives us infinitely many solutions, but the solutions form 3 equivalence classes. \\
All we need is $k = 0, 1, 2$, which gives $\theta_1 = \frac{\pi}{6}$, $\theta_2 = \frac{5\pi}{6}$, $\theta_3 = \frac{3\pi}{2}$ \\
Our solutions partitioned the unit circle into 3 equally spaced wedges. \\
The solutions to $w^3 = i$ are $w_1 = \frac{\sqrt3}{2} + \frac{1}{2}i$, $w_2 = -\frac{\sqrt3}{2} + \frac{1}{2}i$ and $w_3 = -i$. \\
This problem of unity can be extended to solving $w^k = z$ for $k \in \mathbb{N}$, $z \in \mathbb{C}$ for unknown k-solutions w.  


\newpage
\section{Stereographic Projections, Exponentials and Logs}
\subsection{Stereographic Projections} 
We can express the complex plane on the unit sphere in $\mathbb{R}^3$. To perform this we project points on the surface of the sphere along the line from the North Pole $(0, 0, 1)$ through the point and onto the plane $z = 0, \mathbb{C}$ \\ 
$p_1 = (x_1, x_2, x_3) \to z = a + ib = \frac{x_1 + ix_2}{1 - x_3}$ \\
$x_1 = \frac{2a}{|z|^2 + 1}$, $x_2 = \frac{2b}{|z|^2 + 1}$, $x_3 = \frac{|z|^2 - 1}{|z|^2 + 1}$ \\
Points in the northern hemisphere $P_1$, have $|z_1| > 1$; while points in the southern hemisphere $P_2$, have $|z_2| < 1$. 
\subsubsection{Mapping} 
$\mathbb{S}^2 \to \mathbb{C}$ \\ 
$N = (0, 0, 1) \to \infty$ \\
$S = (0, 0, -1) \to 0$ \\ 
lines of latitude $\to$ $|z| = r$, circles \\
lines of longitude $\to$ $\operatorname{Arg}(z) = \pm\theta$, lines through $(0, 0)$ \\
\textbf{Note}: In general, circles on $\mathbb{S}^2$ map to circles and lines in $\mathbb{C}$, orientation is not always preserved. 

\subsection{Complex Logarithm}
\subsubsection{Logarithm of Real Numbers}
Anytime we are dealing with power, the log function is very useful. 
\begin{equation}
\log{x} = \int_{1}^{x} \frac{1}{t} dt \mbox{ for } x \in \mathbb{R}
\end{equation}
$$\frac{d}{dx}x^x = \frac{d}{dx}e^{\ln{x^x}} = \frac{d}{dx}e^{x\ln x} = e^{x\ln x}(x \cdot \frac{1}{x} + \ln{x}) = x^x(1 + \ln{x})$$

\subsubsection{Logarithm of Complex Numbers} 
Remember from Euler's Formula, $e^{i\theta}= \cos\theta + i\sin\theta$. \\
$e^z = e^{a + ib} = e^ae^{ib}$ \\
$\operatorname{Arg}(e^z) = b$, $|e^z| = e^a > 0$ \\
Therefore, if $a$ is held fixed, $e^z$ maps to a circle as $b$ changes. \\
On the other hand, if $b$ is held fixed, $e^z$ maps to a line through $(0, 0)$. 

\subsubsection{Derivation of Complex Logarithm}
We want $e^{\log{(z)}} = z$ for all $z \neq 0$, and thus 
$$e^{\Re(\log{(z)}) + i\Im(\log{(z)})} = e^{\Re(\log{(z)})}e^{i\Im(\log{(z)})} = |z|e^{i\theta} = z$$
$$\Rightarrow |z| = e^{\Re(\log{(z)})}$$ 
$$\Rightarrow \Re(\log{(z)}) = \log{|z|}$$
From the imaginary part we find 
$$e^{i\theta} = e^{i\Im(\log{(z)})}$$
$$\Rightarrow \operatorname{arg}(z) = \theta = \Im(\log{(z)})$$
$$\Rightarrow \Im(\log{(z)}) = \operatorname{Arg}(z)$$ 
because $arg(z)$ is not well defined. \\
Our constructed inverse of $e^z$ is a multi-valued function 
$$\log(z) = \log|z| + i\operatorname{arg}(z)$$

\subsubsection{Conclusion from Derivation}
$$\log(z) = \log|z| + i\operatorname{arg}(z)$$
$$\operatorname{Log} (z) = \log|z| + i\operatorname{Arg}(z)$$
\textbf{Note}: $\operatorname{Log}(z)$ does not have all the nice behavior as $\mathbb{R}$-valued $\log(x)$: $\operatorname{Log}{(z^k)}$. \\
Sometimes they are co-terminal angles, but they are not equal. See the following example: \\
\[ \begin{cases} 
      \operatorname{Log}(i^3) = \operatorname{Log}(-i) = -i\frac{\pi}{2} \\
      3\operatorname{Log}(i) = 3 \cdot (i\frac{\pi}{2}) = i\frac{3\pi}{2}
   \end{cases}
\]
\textbf{Example}: Compute $3^i$: \\
$$3^i = e^{\operatorname{Log}{3^i}} = e^{i\operatorname{Log}{3}} = \cos{(\operatorname{Log}{3})} + i \sin{(\operatorname{Log}{3})}$$

\subsubsection{How Logarithm acts on curves}
\[ \begin{cases} 
      \mbox{Maps a circle with radius } r \mbox{ to a vertical line passing through } (\ln(r), 0)\\
      \mbox{Maps a line with angle } \theta \mbox{ passing through the origin to a horizontal line passing through } (0, i\theta)
   \end{cases}
\]

\newpage
\section{Topology in $\mathbb{C}$}
\subsection{Complex Sequence}
Let $\{Z_n\}$ be a sequence in $\mathbb{C}$. 
\subsubsection{Cauchy Sequence}
The sequence is Cauchy if for all $\epsilon > 0$, there is a $N \in \mathbb{N}$ such that for all $n, m > N$, $|z_n - z_m| < \epsilon$. 
\subsubsection{Sequence Convergence}
The sequence converges if $|z_n - z| \to 0$ as $n \to \infty$. The distance between $z_n$ and $z$ vanishes. 
\subsubsection{Completeness of $\mathbb{C}$}
$\{z_n\}$ converges if and only if $\{z_n\}$ is Cauchy. \\
\textbf{Proof}:\\
We show this by treating $\mathbb{C}$ as $\mathbb{R}^2$ and exploiting $\{X_n\}$ converges if and only if $\{X_n\}$ is Cauchy. \\
($\Longrightarrow$) (If $z_n \to z$, then $\Re(z_n) \to \Re(z)$ and $\Im(z_n) \to \Im(z)$. Since the sequences of $\mathbb{R}^2$ converge, they are Cauchy. \\
$|Z_n - Z_m| \leqslant |\Re(Z_n - Z_m)| +  |\Im(Z_n - Z_m)| = |\Re(Z_n) - \Re(Z_m)| + |\Im(Z_n) - \Im(Z_m)|$ \\
Upper bounds can be picked to be less than $\frac{\epsilon}{2}$ for some $N$. Therefore, $|Z_n - Z_m| \to 0$. \\
\newline 
($\Longleftarrow$) If $\{Z_n\}$ is Cauchy, so are $\{\Re(Z_n)\}$ and $\{\Im(Z_n)\}$. But these are $\mathbb{R}$-sequences that converge. Therefore, $\{Z_n\}$ converges. 

\subsection{Complex Set} 
Let $\Omega \subset \mathbb{C}$. Sets can be open, closed, both, or neither. 
\subsubsection{Open Set} 
If for any $z_0 \in \mathbb{C}$, there exist some $\epsilon > 0$, such that the set $B_\epsilon(z_0) = \{z||z - z_0| < \epsilon\}$ is contained in $\Omega$, then $\Omega$ is open. \\
$\Omega$ is open if and only if $\Omega^c$ is closed. \\ 
$\Omega$ is open if and only if $\Omega$ is equal to its own interior, which means it does not contain its boundary points $\partial \Omega$, i.e. it does not contain its closure. 
\subsubsection{Closed Set} 
If $\Omega$ contains its limit point, then $\Omega$ is closed. \\
$\Omega$ is closed if and only if $\Omega^c$ is open. \\ 
$\Omega$ is closed if and only if $\Omega$ contains its boundary points. 
\subsubsection{Compact Set} 
If $\Omega$ can be contained in a disk of finite radius, then $\Omega$ is bounded. 
\subsubsection{Compact Set} 
If $\Omega$ is closed and bounded, then $\Omega$ is compact. This resembles $[a, b]$ in $\mathbb{R}$. 
\subsubsection{Connected Set} 
If any two points in $\Omega$ can be connected by a path, then $\Omega$ is connected. \\
\underline{Simply Connected Set}: A simply connected set has no "holes" in it. For example, $\Omega = \{z||z - c|<4\}$. \\
A connected but not simply connected set is an annulus, $\Omega = \{z|2< |z - c|<4\}$

\subsubsection{Boundary of Set}
The boundary of $\Omega$, $\partial \Omega$ is all points with $\epsilon$-balls intersecting $\Omega$ and $\Omega^c$ for all $\epsilon > 0$. 
\subsubsection{Interior of Set} 
The interior of $\Omega$, Int$(\Omega)$, is all points in $\Omega$ with a $\epsilon$-ball contained in $\Omega$ for some $\epsilon > 0$. "Largest open set in $\Omega$". 
\subsubsection{Closure of Set} 
The closure of $\Omega$ is the union of $\Omega$ and its boundary $\partial \Omega$. 
\subsubsection{Domain}
If a set is open and connected in $\mathbb{C}$, it is a domain. \\
A domain can be traversed by a path of horizontal and vertical line segments. 

\subsubsection{Practice Examples} 
Determine whether the following sets are open or closed. 
\begin{enumerate}
  \item $\Omega = \mathbb{C} \backslash \{0\}$ \\
  $\Omega$ is open since it does not contain its closure, the point 0. \\
  $\Omega$ is not closed since it does not contain its limit points. Let $z_n = \frac{1}{n}$. Then $z_n = \frac{1}{n} \to 0 \notin \Omega$. \\
  Therefore, $\Omega$ is open. 
  \item $\Omega = \{z||z| \geqslant 1\}$ \\
  $\Omega$ is not open since any $\epsilon$-ball at 1 intersects $\Omega^c$. \\
  $\Omega$ is closed since $\Omega^c$ is open. \\
  Therefore, $\Omega$ is closed. 
  \item $\Omega = \{z||z| > 1\}$ \\
  $\Omega$ is open since $\Omega^c$ is closed. \\
  $\Omega$ is not closed since it does not contain its limit points. Let $z_n = \frac{1}{n} + 1$. Then $z_n = \frac{1}{n} + 1 \to 1 \notin \Omega$.
  \item $\Omega = \mathbb{C} \backslash (0, 1)$ \\ 
  $\Omega$ is not open. Its complement is $[0, 1]$. Even though it is closed in $\mathbb{R}$, it is not closed in $\mathbb{C}$, because any 2D $\epsilon$-ball will always extend outside of the set $z \in (0, i)$. Hence, $\Omega^c$ is not open and not closed. \\
  $\Omega$ is not closed since it does not contain its limit points. Let $z_n = \frac{1}{3} + i\frac{1}{n}$. Then $z_n = \frac{1}{3} + i\frac{1}{n} \to \frac{1}{3} \notin \Omega$. \\
  Therefore, $\Omega$ is neither open nor closed. 
  \item $\Omega = \mathbb{C} \backslash [0, 1]$
  $\Omega$ is open since $\Omega^c = [0, 1]$ is closed in $\mathbb{C}$. \\ 
  $\Omega$ is not closed since it does not contain its limit points. Let $z_n = \frac{1}{3} + i\frac{1}{n}$. Then $z_n = \frac{1}{3} + i\frac{1}{n} \to \frac{1}{3} \notin \Omega$. \\
  Therefore, $\Omega$ is open. \\ 
  Note: $\Omega^c$ is not open in $\mathbb{C}$. 
\end{enumerate}


\newpage
\section{Continuity and Branch Cuts}
\subsection{Complex Continuity} 
Let $f:\Omega \to \mathbb{C}$, $\Omega$ is open and connected. If $z_n \to z_0$ implies $f(z_n) \to f(z_0)$, then $f$ is continuous at $z_0$. Also, $f$ is bounded near $z_0$. \\
$f$ is continuous if for every $\epsilon > 0$, there is $\delta > 0$ such that $|z - z_0| < \delta \Rightarrow |f(z) - f(z_0)| < \epsilon$. \\
\newline 
$\bullet$ In either case, $\Re(f(z))$ and $\Im(f(z))$ are each continuous if and only if $f(z)$ is continuous. This follows the pattern as $\mathbb{C}$ being complete. \\
\newline
$\bullet$ If $f$ and $g$ are continuous, then so are $f + g$m $f \times g$ and $\frac{f}{g}$ (provided $g(z) \nrightarrow 0$)

\subsection{Complex Limits}
Just like in $\mathbb{R}^2$, limits are direction independent. Do not restrict limits to just $\Re \to 0$ or $\Im \to 0$. See the following example. \\
\begin{equation*}
\lim_{(x, y) \to (0, 0)} \frac{2x^2y}{x^4 + y^2} \mbox{ does not exist}
\end{equation*} 
as $x \to 0$, $y = 0$, then $f \to 0$, while $y = x^2, x \to 0$, then $f \to 1$.

\subsection{Branch Cuts}
$\operatorname{Log}$, $z^\frac{1}{2}$ and $\arctan(z)$ are constructed by restricting the range of $e^z$, $z^2$ and $\tan(z)$. \\
For example, in creating $\operatorname{Log}(z) = \ln|z| + i\operatorname{Arg}(z)$, we made a choice that $\operatorname{Arg}(z) \in (-\pi, \pi]$, $\operatorname{Arg}(0)$ does not exist. 
\subsubsection{Example of a Branch Cut}
Consider a path around $z_0 \neq 0$, $\gamma(t) = z_0 + re^{it}$. $\theta(t) = \operatorname{arg}(\gamma(t))$ \\
As we traverse the circle, $t \in (-\pi, \pi]$, \\
$$\theta(t) = \operatorname{arg}(\gamma(t)) = \operatorname{Arg}(z_0 + re^{it}) + 2\pi k = \operatorname{Arg}(z_0 + re^{i(t + 2\pi)}) + 2\pi k = \operatorname{arg}(\gamma(t + 2\pi)) = \theta(t + 2\pi)$$
Therefore, the angle $\theta(t)$ changes smoothly for all $t$ and we stay on the same branch of $\operatorname{Arg}(\gamma(t))$. That is to say, the $k \in \mathbb{Z}$ is the same for all $t$.\\
\newline
Compare this with any circular path about $z = 0$, $\gamma_0$. Let $\gamma_0(t) = re^{it}$, $t \in (-\pi, \pi]$. As we traverse the circle once, we have a discontinuity in the principal angle of $\gamma_0(t)$. In particular, $\theta(\gamma_0(t)) \neq \theta(\gamma_0(t + 2\pi))$ \\
$$\theta(t) = \operatorname{arg}(\gamma(t)) = \operatorname{Arg}(re^{it}) + 2\pi k \neq \operatorname{Arg}(re^{i(t + 2\pi)}) + 2\pi (k + 1) = \operatorname{arg}(\gamma(t + 2\pi)) = \theta(t + 2\pi)$$
We jump from the $k$th to the $(k+1)$th branch of $\operatorname{Arg}$. Therefore, $\operatorname{Arg}(z)$ has a branch point at $z = 0$. \\

\subsubsection{Definition of Branch Cuts and Branch Points}
If every neighborhood of $z_0$ contains a path $\gamma(t)$ around $z_0$ that leads to a jump discontinuity in $f$, then $z_0$ is a branch point of $f(z)$. \\ 
$\bullet$ At this point, it suffices to study paths of the form $\gamma(t) = z_0 + re^{it}$ for $t \in (-\pi, \pi)$, and see if $f(\gamma(t)) = f(\gamma(t + 2\pi))$ holds for all $t$. \\
$\bullet$ $\operatorname{Arg}$ is discontinuous for all $x$ on the negative $\mathbb{R}$-axis, $\mathbb{R}^-$. We call this the principal branch cut of the multi-valued function $\operatorname{arg}$. Specifically, \\
$\operatorname{Arg}(\gamma_0(t)) \to \pi$ as $t \to \pi^-$ \\
$\operatorname{Arg}(\gamma_0(t)) \to -\pi$ as $t \to -\pi^+$ \\
but $\gamma_0(\pi) = \gamma_0(-\pi)$ since $\pi$ and $-\pi$ are coterminal. \\
$\mathbb{R}^-$ is the principal branch of $\operatorname{Log}$, $\operatorname{Arg}$, and $z^{\frac{1}{2}}$. \\
$\bullet$ The endpoints of a branch cut are branch points, $\operatorname{Arg}$ has 0 and $\infty$ as its branch points. 

\newpage
\section{Differentiability in $\mathbb{C}$}
Let $f: \Omega \to \mathbb{C}$ for some domain $\Omega$. Then $f$ is differentiable at $z_0$ if the following exists. $$\frac{d}{dz}f(z)|_{z = z_0} = f'(z_0) = \lim_{h\to0}\frac{f(z + h) - f(z)}{h}$$
This limit must exist on all paths to $z_0$, since $h \in \mathbb{C}$. We could also take $z_n \to z_0$ and use $\frac{f(z_0) - f(z_n)}{z_0 - z_n} \to f'(z_0)$. 
Remember limits are computed by looking at the difference in the modulus, $|\frac{f(z_0) - f(z_n)}{z_0 - z_n} - f'(z_0)| \to 0$ as $n \to \infty$. \\
If $f'(z_0)$ exists on all points $z_0 \in \Omega$, open and connected in $\mathbb{C}$, then $f$ is holomorphic/$\mathbb{C}$-differentiable/analytic on $\Omega$. The connection between $\mathbb{R}$ and $\mathbb{C}$ analytic will be clear when we cover $\mathbb{C}$-power series. \\
If $f'(z)$ exists everywhere in $\mathbb{C}$, then $f$ is an entire/meromorphic function.  

\subsection{Difference between $\mathbb{R}$ and $\mathbb{C}$ differentiability}
$\bullet$ $f: \mathbb{R} \to \mathbb{R}$ 
$$\lim_{h\to0}\frac{f(x+h) - f(x)}{h} = f'(x)$$ 
has only two paths to $x$, namely $h \to 0^+$ and $h \to 0^-$. \\ 
Tangent plane or linear approximation: 
$$f(x) \approx f(a) + f'(a)(x - a)$$
$\bullet$ $f: \mathbb{R}^2 \to \mathbb{R}$
$$\lim_{h\to0}\frac{f(x+h, y) - f(x, y)}{h} = f_x$$ 
is also a 1D limit and a partial derivative. \\
Tangent plane or linear approximation: 
$$f(x,y) \approx f(a,b) + f_x(a,b)(x - a) + f_y(a, b)(y - b)$$
$\bullet$ $f: \mathbb{R}^2 \to \mathbb{R}^2$ $f(x, y) = (u(x, y),v(x,y))$ \\
Then $f$ is differentiable if the Jacobian Matrix
\begin{equation*}
J(f) = 
\begin{bmatrix}
 u_x & u_y \\
 v_x & v_y
\end{bmatrix}
\end{equation*}
can approximate the local change in $f$. \\
In each of the cases above, we are only measuring change in a few directions. However, $h \to0$ in $\mathbb{C}$ can be from any direction in 2-space. Therefore, $f'(z)$ existing is a much stronger condition for $f$ on $\mathbb{C}$ than on $\mathbb{R}$. Consider the following example: \\
Let $g(z) = \Re(z) = \frac{z + \conj{z}}{2}$, which is a linear combination of continuous functions. Assume $h \in \mathbb{R}$, 
$\frac{g(z + ih) - g(z)}{h} = \frac{\Re(z) - \Re(z)}{h} \to 0$ as $h \to 0$ \\
Compare this with $\frac{g(z + ih) - g(z)}{h} = \frac{\Re(z) + h -\Re(z)}{h} = 1 \to 1$ as $h \to 0$. Therefore, the function is nowhere differentiable in $\mathbb{C}$. The problem with $g'(z)$ had to do with $\conj{z}$, despite reflection in $\mathbb{R}^2$ about $y = 0$ is differentiable. We will discover conditions on $u_x$, $u_y$, $v_x$, and $v_y$ that ensure $f'$ exists for $f(z) = u(x, y) + iv(x, y)$ in the next chapter.  \\
If $f$ is differentiable on $\Omega$, it is continuous on $\Omega$. \\
The power rule holds too: $\frac{d}{dz}z^n = nz^{z - 1}$\\
As does the product, quotient, L'Hospital's and chain rule. In fact, most old results hold as well. \\
If $f(z) = \operatorname{Log}(z)$, then $f'(z) = \frac{1}{z}$ \\
If $f(z) = \tan^{-1}(z)$, then $f'(z) = \frac{1}{z^2 + 1} = \frac{1}{(z + i)(z - i)}$, which does not exist for $z = \pm i$



\newpage
\section{The Cauchy Riemann equations}
\section{Harmonic Functions}
\section{Conformal Maps}
\section{Bilinear Transformations}
\section{Contour Integral in $\mathbb{C}$}
\section{Cauchy's Closed Curve Theorem and the Fundamental Theorem of Calculus}
\section{Cauchy's Integral Formula}
\section{Growth Conditions of Holomorphic Functions}
\section{Convergence of Infinite Series in $\mathbb{C}$}
\section{Power Series in $\mathbb{C}$}
\section{Series Expansion of Holomorphic Functions}
\section{Open Mapping Theorem and Reflection Principle}
\section{Laurent Series}
\section{Residue Theorem}
\section{Improper Integrals}
\section{Argument Principle and Rouche's Theorem}

\iffalse
Introduction to Myself
\subsection{Subsection}
Education Background
\subsubsection{Subsubsection} 
More Text
\paragraph{Paragraph}
Some more text 
\subparagraph{Subparagraph} 
Even more text 
\section{Another Section} 
\fi

\newpage
\title{Chapter 1: Algebra in C}

 
\begin{equation*}
	f(x) = x^2
\end{equation*}
this formula is an example $f(x) = x$
\begin{align*}
1 + 2 &= 3\\
1 &= 3 - 2
\end{align*}
\begin{align*}
f(x) &= x ^ 2\\
g(x) &= \frac{1}{x}\\
h(x) &= \int^a_b \frac{1}{x}x^3\\
F(x) &= \frac{1}{\sqrt{x}}
\end{align*}

\begin{align*}
\left[
\begin{matrix}
1 & 0\\
0 & 1
\end{matrix}
\right] \\
\left(\frac{1}{\sqrt{x}}\right)
\end{align*}

\newpage
Core Material:
1. Finding patterns in data; using them to make predictions. 
2. Models and statistics help us understand patterns. 
3. Optimization algorithms "learn" the patterns.

Classification: 
1. 

\end{document}