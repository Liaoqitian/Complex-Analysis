\documentclass[11pt]{article}
\usepackage{amsmath} 
\usepackage{graphicx}
\usepackage{subcaption}
\usepackage{sectsty}
\usepackage{amssymb}
 \usepackage{lipsum}
\usepackage{titlesec}
\usepackage{romannum}
\usepackage{enumitem}
\usepackage{mathtools}
\usepackage[super]{nth}
\usepackage{tikz}
\usepackage{ragged2e}
\newcommand*\circled[1]{\tikz[baseline=(char.base)]{
            \node[shape=circle,draw,inner sep=2pt] (char) {#1};}}

%Antiderivative Line
\DeclareMathOperator{\di}{d\!}
\newcommand*\Eval[3]{\left.#1\right\rvert_{#2}^{#3}}

\setlist[itemize,1]{leftmargin=\dimexpr 26pt-.5in}
%Set Equation Space above and Below
\setlength{\abovedisplayskip}{2pt}
\setlength{\belowdisplayskip}{2pt}


\sectionfont{\fontsize{12}{15}\selectfont}
\title{Introduction to Complex Analysis}
\author{Qitian Liao}
\date{Aug 3, 2020} 
\usepackage[left=2cm, right=2cm, top=2cm]{geometry}
\setlength\parindent{0pt}

\DeclarePairedDelimiter\abs{\lvert}{\rvert}
\DeclarePairedDelimiter\norm{\lVert}{\rVert}

\begin{document}
%\pagenumbering{gobble}
\maketitle
\newpage
\tableofcontents
\def\Arg{\mathop{\operator@font Arg}\nolimits}
%\newpage
\pagenumbering{arabic}
\titleformat*{\section}{\Large\bfseries}
\titleformat*{\subsection}{\large\bfseries}
\titleformat*{\subsubsection}{\normalsize\bfseries}
\titleformat*{\paragraph}{\large\bfseries}
\titleformat*{\subparagraph}{\large\bfseries}

\titlespacing\section{0pt}{5pt plus 4pt minus 2pt}{5pt plus 2pt minus 2pt}
\titlespacing\subsection{0pt}{10pt plus 4pt minus 2pt}{5pt plus 2pt minus 2pt}
\titlespacing\subsubsection{0pt}{5pt plus 4pt minus 2pt}{5pt plus 2pt minus 2pt}

\newpage
\section{Algebra of the Complex Plane}
\subsection{Introduction to Complex Numbers}
Let $z = a + ib \in \mathbb{C}$ where $a, b \in \mathbb{R}$ and $i ^ 2 = -1$. \\
This number can be thought of as a point in 2-space, $\mathbb{R} ^ 2$, $(a, b)$ or as a position in $\mathbb{C}$. \\ 
$\mathbb{R} ^ 2$: $\oplus$ addition; $\odot$ scalar multiplication. \\
$\mathbb{C}$ : $\oplus$ addition; $\odot$ scalar multiplication; a vector space; have multiplication of elements, $\mathbb{C}$ is a field. \\
\begin{equation*} 
\mbox{If } z = a + ib \mbox{, } w = c + id \mbox{, then }zw = (ac - bd) + i(ad + cb)
\end{equation*}
$$zw = wz$$
$$z(w + \alpha) = zw + z\alpha$$ 
$$(zw)\alpha = z(w\alpha)$$ 

\subsection{Conjugate of Complex Numbers} 
\newcommand*\conj[1]{\overline{#1}}
\textbf{Definition of Conjugate.} \\
The complex conjugate of $z$, $\conj{z}$, is defined by 
$$\conj{z} = a - ib$$ 
Geometric representation: The image of $\bar z$ is the reflection of z about the Real axis. \\
\newline
\textbf{Properties of Conjugate.} \\
$$\conj{\conj{z}} = z$$
$$\conj{zw} = \conj{z}\conj{w}$$
$$\conj{z + w} = \conj{z} + \conj{w}$$
\begin{equation*}
\overline{z} = z \mbox{ if and only if } z \in \mathbb{R}
\end{equation*}
\textbf{Real and Imaginary Parts.} \\
\newline
\noindent We can project $z$ onto the Real or Imaginary axis and measure its distance from 0: 
$$\Re(z) = a$$
\begin{equation*}
\Im(z) = b \mbox{, not } ib
\end{equation*}

Each function is a map $\mathbb{C} \to \mathbb{R}$. Then 
$$\Re(z) = \frac{z + \conj{z}}{2}$$
$$\Im(z) = \frac{z - \conj{z}}{2i}$$
This is similar to the pattern with even/odd functions. 

\subsection{Modulus of Complex Numbers.}
$z\conj{z} = (a + ib)(a - ib) = a^2 + b^2 \in \mathbb{R}$  \\

\textbf{Definition of Modulus.} \\
$|z|$ length/modulus of $z$ is defined by:
$$|z| = (a^2 + b^2)^{\frac{1}{2}} = (z\conj{z})^{\frac{1}{2}} \in \mathbb{R}$$
\newline
\textbf{Properties of Modulus.} \\
$$|zw| = |z||w|$$ 
$$|z| = |\conj{z}|$$  
$$|z| \geqslant 0$$
\begin{equation*}
|z| = 0 \mbox{ if and only if } z = 0
\end{equation*}
\newline
\textbf{Triangle Inequality and Reverse Triangle Inequality.} \\

\[ \begin{cases} 
      |z + w| \leqslant |z| + |w| \\
      |z| - |w| \leqslant |z - w|
   \end{cases}
\]

$$z = z - w + w \Rightarrow |z| = |z - w + w| \Rightarrow |z| \leqslant |z - w| + |w| \Rightarrow |z| - |w| \leqslant |z - w|$$. 
\newline
\textbf{Complex Division.} \\
With $z\conj{z} \in \mathbb{R}$, we can define complex division by reducing it to a multiplication problem. 
$$\frac{z}{w} = \frac{z\conj{w}}{w\conj{w}} = \frac{1}{w\conj{w}}(z\conj{w})$$ 
We also have 
\begin{equation*} 
\abs[\Big]{\frac{z}{w}} = \frac{|z|}{|w|} \mbox{ for } w \neq 0 
\end{equation*}

\textbf{Distance in the plane.} \\
A disk in the complex plane centered at $c$ of radius $r \in \mathbb{R}$ is of the form 
$$\{z \in \mathbb{C} \mid |z - c| \leqslant r\}$$
\\
\subsection{Complex Polynomial}
A complex polynomial $p(z)$ of degree n is of the form: 
$$p(z) = a_nz^n + a_{n - 1}z^{n - 1} + \cdots + a_1z + a_0$$
where $a_n \neq 0$ and $a_i \in \mathbb{C}$ for $ i = 0, \cdots , n$ \\
\newline
\textbf{Fundamental Theorem of Algebra.} \\

The factorization of $p(z)$ factors over $\mathbb{C}$ is unique, 
$$p(z) = c(z - z_1)^{m_1}...(z - z_k)^{m_k}$$
We have roots $z_i \in \mathbb{C}$ of $p(z)$ with order $m_i \in \mathbb{N}$. \\
For example, if $p(z) = z^2 + 4 = (z + 2i)(z - 2i)$, then it factors over $\mathbb{C}$ but not $\mathbb{R}$. \\
\textbf{Note}: 
$\mathbb{C}$ is an algebraically closed field, there are no irreducible polynomials in $\mathbb{C}$. \\
\textbf{Note}: 
$\mathbb{R}$,  $\mathbb{Q}$, $\mathbb{Z}$, $\mathbb{N}$ are not algebraically closed. \\

\newpage
\section{Geometry of the Complex Plane}
\subsection{Properties of Polar Forms} 
Complex numbers can be represented in polar forms: 
$$z = |z|(\cos\theta + i\sin\theta)$$
with modulus $|z|$ and argument $\theta$. To change between the coordinate systems it follows: 
$$|z| = (a^2 + b^2)^{\frac{1}{2}}$$
$$\tan\theta = \frac{b}{a}$$
$$a = |z|\cos\theta= \Re(z)$$
$$b = |z|\sin\theta = \Im(z)$$
Note that $\theta_R = \arctan(\frac{b}{a})$ is a reference angle of $z$. To find $\theta$ from $\theta_R$, you need to consider the signs of $a$ and $b$. \\
\newline
\textbf{Example.} \\
$$z = -3 + 3i = 3\sqrt2(\cos\frac{3\pi}{4} + \sin\frac{3\pi}{4})$$
$$\theta_R = \arctan(\frac{3}{-3})= -\frac{\pi}{4}$$
\begin{equation*}
\theta = \pi + \theta_R = \pi -\frac{\pi}{4} = \frac{3\pi}{4} \mbox{, since } \theta \mbox{ is in } \Romannum{2} .
\end{equation*}
\subsection{Definition of Argument and argument}
$\operatorname{Arg}(z)$ is $z$'s principle polar angle $\theta$, $z \neq 0$, where $\theta \in (-\pi, \pi]$. \\
$\operatorname{arg}(z)$ is all of $z$'s polar angles, $\theta + 2k\pi$, $k \in \mathbb{Z}$. 

\subsection{Euler's Formula} 
Euler's Formula is defined as a linear combination of $\cos\theta$ and $\sin\theta$, $\mathbb{R}$-valued functions. 
$$ e^{i\theta}= \cos\theta + i\sin\theta $$
It allows us to express $z$ in polar form by 
$$ z = |z|e^{i\theta}$$
-1 has polar angle $\pi$ and modulus 1, 
\begin{equation*}
-1 = e^{i\pi} \mbox{ or } e^{i\pi} + 1 = 0
\end{equation*}
By the angle addition formulas from trigonometry we find: 
$$e^{i\theta}e^{i\varphi} = e^{i(\theta + \varphi)}$$
$$(e^{i\theta})^k = e^{i\theta k}$$

\subsection{Geometric Understanding of Multiplication}
The polar angle of $zw$ is the sum of the polar angles of $z$ and $w$. The modulus is the product of the moduli. 
$$ \operatorname{Arg}(zw) = \operatorname{Arg}(z) + \operatorname{Arg}(w)$$
$$ \operatorname{Arg}(\conj{z}) = -\operatorname{Arg}(z)$$
Question: How about $\frac{z}{w}$ and $z^4$? \\
It follows from trigonometry that $|e^{i\theta}| = 1$, if $\theta \in (-\pi, \pi]$ we get a parametrization of the unit circle. \\
\newline
\textbf{Example.} Discover all solutions to $w^3 = i = z$ \\
Let $p(z) = w^3 - i$. By Fundamental Theorem of Algebra, there are 3 roots of $p(z)$. \\
Therefore, $3\theta= \frac{\pi}{2} + 2\pi k$, $k \in \mathbb{Z}$ \\
This gives us infinitely many solutions, but the solutions form 3 equivalence classes. \\
All we need is $k = 0, 1, 2$, which gives $\theta_1 = \frac{\pi}{6}$, $\theta_2 = \frac{5\pi}{6}$, $\theta_3 = \frac{3\pi}{2}$ \\
Our solutions partitioned the unit circle into 3 equally spaced wedges. \\
The solutions to $w^3 = i$ are $w_1 = \frac{\sqrt3}{2} + \frac{1}{2}i$, $w_2 = -\frac{\sqrt3}{2} + \frac{1}{2}i$ and $w_3 = -i$. \\
This problem of unity can be extended to solving $w^k = z$ for $k \in \mathbb{N}$, $z \in \mathbb{C}$ for unknown k-solutions w.  


\newpage
\section{Stereographic Projections, Exponentials and Logs}
\subsection{Stereographic Projections} 
We can express the complex plane on the unit sphere in $\mathbb{R}^3$. To perform this we project points on the surface of the sphere along the line from the North Pole $(0, 0, 1)$ through the point and onto the plane $z = 0, \mathbb{C}$ \\ 
$$p_1 = (x_1, x_2, x_3) \to z = a + ib = \frac{x_1 + ix_2}{1 - x_3}$$
$$x_1 = \frac{2a}{|z|^2 + 1}$$
$$x_2 = \frac{2b}{|z|^2 + 1}$$
$$x_3 = \frac{|z|^2 - 1}{|z|^2 + 1}$$
Points in the northern hemisphere $P_1$, have $|z_1| > 1$. \\
Points in the southern hemisphere $P_2$, have $|z_2| < 1$. \\
\newline
\textbf{Mapping from Stereographic Space to the Complex Plane.} 
\begin{align*}
\mathbb{S}^2 &\to \mathbb{C}\\
N = (0, 0, 1) &\to \infty \\
S = (0, 0, -1) &\to 0 \\
\mbox{lines of latitude } &\to |z| = r \mbox{, circles} \\
\mbox{lines of longitude } &\to \operatorname{Arg}(z) = \pm\theta \mbox{, lines through } (0, 0) 
\end{align*}
Note that in general, circles on $\mathbb{S}^2$ map to circles and lines in $\mathbb{C}$, orientation is not always preserved. 

\subsection{Complex Logarithm}
\textbf{Logarithm of Real Numbers.} \\
Anytime we are dealing with power, the log function is very useful. 
\begin{equation*}
\log{x} = \int_{1}^{x} \frac{1}{t} dt \mbox{ for } x \in \mathbb{R}
\end{equation*}
$$\frac{d}{dx}x^x = \frac{d}{dx}e^{\ln{x^x}} = \frac{d}{dx}e^{x\ln x} = e^{x\ln x}(x \cdot \frac{1}{x} + \ln{x}) = x^x(1 + \ln{x})$$
\newline
\textbf{Logarithm of Complex Numbers.} \\
Remember from Euler's Formula, $e^{i\theta}= \cos\theta + i\sin\theta$. 
$$e^z = e^{a + ib} = e^ae^{ib}$$ 
$$\operatorname{Arg}(e^z) = b$$
$$|e^z| = e^a > 0$$ 
Therefore, if $a$ is held fixed, $e^z$ maps to a circle as $b$ changes. \\
On the other hand, if $b$ is held fixed, $e^z$ maps to a line through $(0, 0)$. \\
\newline
\textbf{Derivation of Complex Logarithm.} \\
We want $e^{\log{(z)}} = z$ for all $z \neq 0$, and thus 
$$e^{\Re(\log{(z)}) + i\Im(\log{(z)})} = e^{\Re(\log{(z)})}e^{i\Im(\log{(z)})} = |z|e^{i\theta} = z$$
$$\Rightarrow |z| = e^{\Re(\log{(z)})}$$ 
$$\Rightarrow \Re(\log{(z)}) = \log{|z|}$$
From the imaginary part we find 
$$e^{i\theta} = e^{i\Im(\log{(z)})}$$
$$\Rightarrow \operatorname{arg}(z) = \theta = \Im(\log{(z)})$$
$$\Rightarrow \Im(\log{(z)}) = \operatorname{Arg}(z)$$ 
because $\operatorname{arg}(z)$ is not well defined. \\
Our constructed inverse of $e^z$ is a multi-valued function 
$$\log(z) = \log|z| + i\operatorname{arg}(z)$$
\newline
\textbf{Conclusion from Derivation.}\\
$$\log(z) = \log|z| + i\operatorname{arg}(z)$$
$$\operatorname{Log} (z) = \log|z| + i\operatorname{Arg}(z)$$
\textbf{Note}: $\operatorname{Log}(z)$ does not have all the nice behavior as $\mathbb{R}$-valued $\log(x)$: $\operatorname{Log}{(z^k)}$. \\
Sometimes they are co-terminal angles, but they are not equal. See the following example: \\
\[ \begin{cases} 
      \operatorname{Log}(i^3) = \operatorname{Log}(-i) = -i\frac{\pi}{2} \\
      3\operatorname{Log}(i) = 3 \cdot (i\frac{\pi}{2}) = i\frac{3\pi}{2}
   \end{cases}
\]
\textbf{Example.} Compute $3^i$: \\
$$3^i = e^{\operatorname{Log}{3^i}} = e^{i\operatorname{Log}{3}} = \cos{(\operatorname{Log}{3})} + i \sin{(\operatorname{Log}{3})}$$
\newline
\textbf{How Logarithm acts on curves. }\\
\[ \begin{cases} 
      \mbox{Maps a circle with radius } r \mbox{ to a vertical line passing through } (\ln(r), 0)\\
      \mbox{Maps a line with angle } \theta \mbox{ passing through the origin to a horizontal line passing through } (0, i\theta)
   \end{cases}
\]

\newpage
\section{Topology in $\mathbb{C}$}
\subsection{Sequence}
Let $\{Z_n\}$ be a sequence in $\mathbb{C}$. \\
\newline
\textbf{Cauchy Sequence. }\\
The sequence is Cauchy if for all $\epsilon > 0$, there is a $N \in \mathbb{N}$ such that for all $n, m > N$, $|z_n - z_m| < \epsilon$. \\
\newline
\textbf{Convergence of Sequence.}\\ 
The sequence converges if $|z_n - z| \to 0$ as $n \to \infty$. The distance between $z_n$ and $z$ vanishes. \\
\newline
\textbf{Completeness of $\mathbb{C}$.} \\
$\{z_n\}$ converges if and only if $\{z_n\}$ is Cauchy. \\
\textbf{Proof.} 
We show this by treating $\mathbb{C}$ as $\mathbb{R}^2$ and exploiting $\{X_n\}$ converges if and only if $\{X_n\}$ is Cauchy. \\
($\Longrightarrow$) (If $z_n \to z$, then $\Re(z_n) \to \Re(z)$ and $\Im(z_n) \to \Im(z)$. Since the sequences of $\mathbb{R}^2$ converge, they are Cauchy. \\
$|Z_n - Z_m| \leqslant |\Re(Z_n - Z_m)| +  |\Im(Z_n - Z_m)| = |\Re(Z_n) - \Re(Z_m)| + |\Im(Z_n) - \Im(Z_m)|$ \\
Upper bounds can be picked to be less than $\frac{\epsilon}{2}$ for some $N$. Therefore, $|Z_n - Z_m| \to 0$. \\
\newline 
($\Longleftarrow$) If $\{Z_n\}$ is Cauchy, so are $\{\Re(Z_n)\}$ and $\{\Im(Z_n)\}$. But these are $\mathbb{R}$-sequences that converge. Therefore, $\{Z_n\}$ converges. 

\subsection{Complex Set} 
Let $\Omega \subset \mathbb{C}$. Sets can be open, closed, both, or neither. \\
\newline
\textbf{Open Set.} \\ 
If for any $z_0 \in \mathbb{C}$, there exist some $\epsilon > 0$, such that the set $B_\epsilon(z_0) = \{z||z - z_0| < \epsilon\}$ is contained in $\Omega$, then $\Omega$ is open. \\
$\Omega$ is open if and only if $\Omega^c$ is closed. \\ 
$\Omega$ is open if and only if $\Omega$ is equal to its own interior, which means it does not contain its boundary points $\partial \Omega$, i.e. it does not contain its closure. \\
\newline
\textbf{Closed Set.} \\ 
If $\Omega$ contains its limit point, then $\Omega$ is closed. \\
$\Omega$ is closed if and only if $\Omega^c$ is open. \\ 
$\Omega$ is closed if and only if $\Omega$ contains its boundary points. \\
\newline
\textbf{Compact Set.}\\ 
If $\Omega$ can be contained in a disk of finite radius, then $\Omega$ is bounded. \\
\newline
\textbf{Compact Set.} \\ 
If $\Omega$ is closed and bounded, then $\Omega$ is compact. This resembles $[a, b]$ in $\mathbb{R}$. \\
\newline
\textbf{Connected Set.}\\ 
If any two points in $\Omega$ can be connected by a path, then $\Omega$ is connected. \\
\underline{Simply Connected Set}: A simply connected set has no "holes" in it. For example, $\Omega = \{z||z - c|<4\}$. \\
A connected but not simply connected set is an annulus, $\Omega = \{z|2< |z - c|<4\}$ \\
\newline
\textbf{Boundary of Set.}\\
The boundary of $\Omega$, $\partial \Omega$ is all points with $\epsilon$-balls intersecting $\Omega$ and $\Omega^c$ for all $\epsilon > 0$. \\
\newline
\textbf{Interior of Set.} \\
The interior of $\Omega$, Int$(\Omega)$, is all points in $\Omega$ with a $\epsilon$-ball contained in $\Omega$ for some $\epsilon > 0$. "Largest open set in $\Omega$". \\
\newline
\textbf{Closure of Set.} \\ 
The closure of $\Omega$ is the union of $\Omega$ and its boundary $\partial \Omega$. \\
\newline
\textbf{Domain.} \\
If a set is open and connected in $\mathbb{C}$, it is a domain. \\
A domain can be traversed by a path of horizontal and vertical line segments. \\
\newline
\textbf{Example.}\\
Determine whether the following sets are open or closed. 
\begin{enumerate}
  \item $\Omega = \mathbb{C} \backslash \{0\}$ \\
  $\Omega$ is open since it does not contain its closure, the point 0. \\
  $\Omega$ is not closed since it does not contain its limit points. Let $z_n = \frac{1}{n}$. Then $z_n = \frac{1}{n} \to 0 \notin \Omega$. \\
  Therefore, $\Omega$ is open. 
  \item $\Omega = \{z||z| \geqslant 1\}$ \\
  $\Omega$ is not open since any $\epsilon$-ball at 1 intersects $\Omega^c$. \\
  $\Omega$ is closed since $\Omega^c$ is open. \\
  Therefore, $\Omega$ is closed. 
  \item $\Omega = \{z||z| > 1\}$ \\
  $\Omega$ is open since $\Omega^c$ is closed. \\
  $\Omega$ is not closed since it does not contain its limit points. Let $z_n = \frac{1}{n} + 1$. Then $z_n = \frac{1}{n} + 1 \to 1 \notin \Omega$.
  \item $\Omega = \mathbb{C} \backslash (0, 1)$ \\ 
  $\Omega$ is not open. Its complement is $[0, 1]$. Even though it is closed in $\mathbb{R}$, it is not closed in $\mathbb{C}$, because any 2D $\epsilon$-ball will always extend outside of the set $z \in (0, i)$. Hence, $\Omega^c$ is not open and not closed. \\
  $\Omega$ is not closed since it does not contain its limit points. Let $z_n = \frac{1}{3} + i\frac{1}{n}$. Then $z_n = \frac{1}{3} + i\frac{1}{n} \to \frac{1}{3} \notin \Omega$. \\
  Therefore, $\Omega$ is neither open nor closed. 
  \item $\Omega = \mathbb{C} \backslash [0, 1]$
  $\Omega$ is open since $\Omega^c = [0, 1]$ is closed in $\mathbb{C}$. \\ 
  $\Omega$ is not closed since it does not contain its limit points. Let $z_n = \frac{1}{3} + i\frac{1}{n}$. Then $z_n = \frac{1}{3} + i\frac{1}{n} \to \frac{1}{3} \notin \Omega$. \\
  Therefore, $\Omega$ is open. \\ 
  Note: $\Omega^c$ is not open in $\mathbb{C}$. 
\end{enumerate}


\newpage
\section{Continuity and Branch Cuts}
\subsection{Complex Continuity} 
Let $f:\Omega \to \mathbb{C}$, $\Omega$ is open and connected. If $z_n \to z_0$ implies $f(z_n) \to f(z_0)$, then $f$ is continuous at $z_0$. Also, $f$ is bounded near $z_0$. \\
$f$ is continuous if for every $\epsilon > 0$, there is $\delta > 0$ such that $|z - z_0| < \delta \Rightarrow |f(z) - f(z_0)| < \epsilon$. \\
\newline 
In either case, $\Re(f(z))$ and $\Im(f(z))$ are each continuous if and only if $f(z)$ is continuous. This follows the pattern as $\mathbb{C}$ being complete. \\
\newline
If $f$ and $g$ are continuous, then so are $f + g$m $f \times g$ and $\frac{f}{g}$ (provided $g(z) \nrightarrow 0$)

\subsection{Complex Limits}
Just like in $\mathbb{R}^2$, limits are direction independent. Do not restrict limits to just $\Re \to 0$ or $\Im \to 0$. See the following example. \\
\begin{equation*}
\lim_{(x, y) \to (0, 0)} \frac{2x^2y}{x^4 + y^2} \mbox{ does not exist}
\end{equation*} 
as $x \to 0$, $y = 0$, then $f \to 0$, while $y = x^2, x \to 0$, then $f \to 1$.

\subsection{Branch Cuts}
$\operatorname{Log}$, $z^\frac{1}{2}$ and $\arctan(z)$ are constructed by restricting the range of $e^z$, $z^2$ and $\tan(z)$. \\
For example, in creating $\operatorname{Log}(z) = \ln|z| + i\operatorname{Arg}(z)$, we made a choice that $\operatorname{Arg}(z) \in (-\pi, \pi]$, $\operatorname{Arg}(0)$ does not exist. \\
\newline
\textbf{Example.} 
Consider a path around $z_0 \neq 0$, $\gamma(t) = z_0 + re^{it}$. $\theta(t) = \operatorname{arg}(\gamma(t))$. \\
As we traverse the circle, $t \in (-\pi, \pi]$, \\
$$\theta(t) = \operatorname{arg}(\gamma(t)) = \operatorname{Arg}(z_0 + re^{it}) + 2\pi k = \operatorname{Arg}(z_0 + re^{i(t + 2\pi)}) + 2\pi k = \operatorname{arg}(\gamma(t + 2\pi)) = \theta(t + 2\pi)$$
Therefore, the angle $\theta(t)$ changes smoothly for all $t$ and we stay on the same branch of $\operatorname{Arg}(\gamma(t))$. That is to say, the $k \in \mathbb{Z}$ is the same for all $t$.\\
\newline
Compare this with any circular path about $z = 0$, $\gamma_0$. Let $\gamma_0(t) = re^{it}$, $t \in (-\pi, \pi]$. As we traverse the circle once, we have a discontinuity in the principal angle of $\gamma_0(t)$. In particular, $\theta(\gamma_0(t)) \neq \theta(\gamma_0(t + 2\pi))$ \\
$$\theta(t) = \operatorname{arg}(\gamma(t)) = \operatorname{Arg}(re^{it}) + 2\pi k \neq \operatorname{Arg}(re^{i(t + 2\pi)}) + 2\pi (k + 1) = \operatorname{arg}(\gamma(t + 2\pi)) = \theta(t + 2\pi)$$
We jump from the $k$th to the $(k+1)$th branch of $\operatorname{Arg}$. Therefore, $\operatorname{Arg}(z)$ has a branch point at $z = 0$. \\

\textbf{Definition of Branch Cuts and Branch Points.}\\
If every neighborhood of $z_0$ contains a path $\gamma(t)$ around $z_0$ that leads to a jump discontinuity in $f$, then $z_0$ is a branch point of $f(z)$. \\ 
In order to find branches, at this point, it suffices to study paths of the form $\gamma(t) = z_0 + re^{it}$ for $t \in (-\pi, \pi)$, and see if $f(\gamma(t)) = f(\gamma(t + 2\pi))$ holds for all $t$. \\
\newline
\textbf{Example.} $\operatorname{Arg}$ is discontinuous for all $x$ on the negative $\mathbb{R}$-axis, $\mathbb{R}^-$. \\
We call this the principal branch cut of the multi-valued function $\operatorname{arg}$. Specifically, \\
\begin{equation*} 
\operatorname{Arg}(\gamma_0(t)) \to \pi \mbox{ as } t \to \pi^- 
\end{equation*}
\begin{equation*} 
\operatorname{Arg}(\gamma_0(t)) \to -\pi \mbox{ as } t \to -\pi^+
\end{equation*} 
but $\gamma_0(\pi) = \gamma_0(-\pi)$ since $\pi$ and $-\pi$ are coterminal. \\
$\mathbb{R}^-$ is the principal branch of $\operatorname{Log}$, $\operatorname{Arg}$, and $z^{\frac{1}{2}}$. \\
The endpoints of a branch cut are branch points, $\operatorname{Arg}$ has 0 and $\infty$ as its branch points. 

\newpage
\section{Differentiability in $\mathbb{C}$}
Let $f: \Omega \to \mathbb{C}$ for some domain $\Omega$. Then $f$ is differentiable at $z_0$ if the following exists. $$\frac{d}{dz}f(z)|_{z = z_0} = f'(z_0) = \lim_{h\to0}\frac{f(z + h) - f(z)}{h}$$
This limit must exist on all paths to $z_0$, since $h \in \mathbb{C}$. We could also take $z_n \to z_0$ and use $\frac{f(z_0) - f(z_n)}{z_0 - z_n} \to f'(z_0)$. 
Remember limits are computed by looking at the difference in the modulus, $|\frac{f(z_0) - f(z_n)}{z_0 - z_n} - f'(z_0)| \to 0$ as $n \to \infty$. \\
If $f'(z_0)$ exists on all points $z_0 \in \Omega$, open and connected in $\mathbb{C}$, then $f$ is holomorphic/$\mathbb{C}$-differentiable/analytic on $\Omega$. The connection between $\mathbb{R}$ and $\mathbb{C}$ analytic will be clear when we cover $\mathbb{C}$-power series. \\
If $f'(z)$ exists everywhere in $\mathbb{C}$, then $f$ is an entire/meromorphic function.  

\subsection{Difference between $\mathbb{R}$ and $\mathbb{C}$ differentiability}
$\bullet$ $f: \mathbb{R} \to \mathbb{R}$ 
$$\lim_{h\to0}\frac{f(x+h) - f(x)}{h} = f'(x)$$ 
has only two paths to $x$, namely $h \to 0^+$ and $h \to 0^-$. \\ 
Tangent plane or linear approximation: 
$$f(x) \approx f(a) + f'(a)(x - a)$$
$\bullet$ $f: \mathbb{R}^2 \to \mathbb{R}$
$$\lim_{h\to0}\frac{f(x+h, y) - f(x, y)}{h} = f_x$$ 
is also a 1D limit and a partial derivative. \\
Tangent plane or linear approximation: 
$$f(x,y) \approx f(a,b) + f_x(a,b)(x - a) + f_y(a, b)(y - b)$$
$\bullet$ $f: \mathbb{R}^2 \to \mathbb{R}^2$ $f(x, y) = (u(x, y),v(x,y))$ \\
Then $f$ is differentiable if the Jacobian Matrix
\begin{equation*}
J(f) = 
\begin{bmatrix}
 u_x & u_y \\
 v_x & v_y
\end{bmatrix}
\end{equation*}
can approximate the local change in $f$. \\
In each of the cases above, we are only measuring change in a few directions. However, $h \to0$ in $\mathbb{C}$ can be from any direction in 2-space. Therefore, $f'(z)$ existing is a much stronger condition for $f$ on $\mathbb{C}$ than on $\mathbb{R}$. \\
\newline
\textbf{Complications with $\conj{z}$.}\\
Consider the following example: \\
Let $g(z) = \Re(z) = \frac{z + \conj{z}}{2}$, which is a linear combination of continuous functions. \\
Assume $h \in \mathbb{R}$, 
\begin{equation*}
\frac{g(z + ih) - g(z)}{h} = \frac{\Re(z) - \Re(z)}{h} \to 0 \mbox{ as } h \to 0 
\end{equation*}
Compare this with 
\begin{equation*} 
\frac{g(z + ih) - g(z)}{h} = \frac{\Re(z) + h -\Re(z)}{h} = 1 \to 1 \mbox{ as } h \to 0
\end{equation*}
Therefore, the function is nowhere differentiable in $\mathbb{C}$. The problem with $g'(z)$ had to do with $\conj{z}$, despite reflection in $\mathbb{R}^2$ about $y = 0$ is differentiable. We will discover conditions on $u_x$, $u_y$, $v_x$, and $v_y$ that ensure $f'$ exists for $f(z) = u(x, y) + iv(x, y)$ in the next chapter.  \\

\subsection{Properties of $f'(z)$}
\textbf{Proposition.} If $f$ is differentiable on $\Omega$, it is continuous on $\Omega$. \\
\textbf{Proposition.} The power rule holds too: $\frac{d}{dz}z^n = nz^{z - 1}$\\
As does the product, quotient, L'Hospital's and chain rule. In fact, most old results hold as well. 
In each of the following cases, there are branch points where $f'(z)$ does not exist. \\
1. If $f(z) = \operatorname{Log}(z)$, then $f'(z) = \frac{1}{z}$ \\
2. If $f(z) = \tan^{-1}(z)$, then $f'(z) = \frac{1}{z^2 + 1} = \frac{1}{(z + i)(z - i)}$, which does not exist for $z = \pm i$ \\
3. If $f(z) = z^\frac{1}{2}$, then $f'(z) = \frac{1}{2}z^{-\frac{1}{2}}$ \\

\textbf{Proposition.} Suppose $f(z)$ is holomorphic on $\Omega$. then $g(z) = \conj{f(\conj{z})}$ is holomorphic on $\Omega^{\ast} = \{z|\conj{z} \in \Omega\}$ \\
\textbf{Proof.} Suppose $f$ is holomorphic on $\Omega$. Let $z_0 \in \Omega^{\ast}$ and $z_n \in \Omega^{\ast}$ for all $n$ and $z_n \to z$. Then $\conj{z_n} \to \conj{z_0}$ in $\Omega$ and for $\epsilon > 0$, there is a $N \in \mathbb{N}$ such that for $n > N$. 
$$\abs*{\frac{f(\conj{z_0}) - f(\conj{z_n})}{\conj{z_0} - \conj{z_n}} - f'(\conj{z_0})} < \epsilon$$ 
$$\abs*{\frac{f(\conj{z_0}) - f(\conj{z_n})}{\conj{z_0} - \conj{z_n}} - f'(\conj{z_0})} = \conj{\abs*{\frac{f(\conj{z_0}) - f(\conj{z_n})}{\conj{z_0} - \conj{z_n}} - f'(\conj{z_0})}} = \abs*{\frac{\conj{{f(\conj{z_0}) - f(\conj{z_n})}}}{\conj{\conj{z_0} - \conj{z_n}}} - \conj{f'(\conj{z_0})}} $$
$$ = \abs*{\frac{\conj{{f(\conj{z_0}) - f(\conj{z_n})}}}{z_0 - z_n} - \conj{f'(\conj{z_0})}} = \abs*{\frac{g(z_0) - g(z_n)}{z_0 - z_n} - \conj{f'(\conj{z_0})}} < \epsilon$$
$$\Longrightarrow g'(z_0) = \conj{f'(\conj{z_0})}$$
Therefore, $g$ is holomorphic on $\Omega^\ast$. \\
This proof is different from when we showed $\frac{d}{dz}\conj{z}$ does not exist. Conjugation must be handled with care. 
%\abs[\Big]
\subsection{Geometric behavior of $f'(z)$}
\textbf{Dilation.}\\ 
$w = f(z) \approx f'(a)(z - a) + f(a)$. Small changes in $z$ should give small changes in $w$. \\
The functions $|z|$ and $\operatorname{Arg}(z)$ are continuous on their domains. \\
If $f'(z_0) \neq 0$ and $f'$ exists on $\Omega$, then 
$$ |f'(z_0)| = \abs*{\lim_{h \to 0}\frac{f(z_0 + h) - f(z_0)}{h}} = \lim_{h \to 0} \abs*{\frac{f(z_0 + h) - f(z_0)}{h}} = \lim_{h \to 0}\frac{|f(z_0 + h) - f(z_0)|}{|h|}$$
$$\Longrightarrow |f'(z_0)||h| \approx |f(z + h) - f(z)|$$
The size of $|f'(z_0)|$ tells us how much $f$ is contracting/dilating near $z_0$. \\
\newline
\textbf{Rotation.}\\
$$\operatorname{Arg}(f'(z_0)) = \operatorname{Arg}\left(\lim_{h \to 0}\frac{f(z_0 + h) - f(z_0)}{h}\right) = \lim_{h \to 0}\operatorname{Arg}\left(\frac{f(z_0 + h) - f(z)}{h}\right)$$
$$ = \lim_{h \to 0} \operatorname{Arg}(f(z_0 + h) - f(z_0)) - \operatorname{Arg}(h)$$
$$\Longrightarrow \operatorname{Arg}(f'(z_0)) \approx \operatorname{Arg}(f(z_0 + h) - f(z_0))$$ 
$$\Longrightarrow \operatorname{Arg}(f'(z_0)) + \operatorname{Arg}(h) \approx \operatorname{Arg}(f(z_0 + h) - f(z_0))$$
Therefore, $f$ rotates vectors from $z_0$ to $z_0 + h$ by the angle $\operatorname{Arg}(f'(z_0))$. \\
\newline
\textbf{Conclusion.} \\
In conclusion, $w = f(z) \approx f(z_0) + f'(z_0)(z - z_0) = c + \rho e^{i\theta}(z - z_0)$ \\
\[ \begin{cases} 
c\mbox{: Translation }\\
\rho \mbox{: Dilation }\\
e^{i\theta} \mbox{: rotation about $z_0$ or complex multiplication. }
   \end{cases}
\]

\newpage
\section{The Cauchy Riemann Equations}
\subsection{The Cauchy Riemann Equations}
Let $f(z) = f(x + iy) = u(x, y) + iv(x, y)$, then $f(z)$ is holomorphic implies the Cauchy Riemann Equations: 
\[ \begin{cases} 
	u_x = v_y \\
	u_y = -v_x
   \end{cases}
\]
\textbf{Proof.}
For $f$, which is $\mathbb{C}$-differentiable, the following representation of $f'(z)$ holds for any path to $z$. 
$$f'(z) = \lim_{h \to 0} \frac{f(z + h) - f(z)}{h}$$
\\
Along the path $x + h + iy \to x + iy$, we get 
$$\lim_{h \to 0} \frac{f(z + h) - f(z)}{h} = \lim_{h \to 0} \frac{f(x + h + iy) - f(x + iy)}{h} = f_x = u_x(x,y) + iv_x(x, y)$$
Along the path $x + iy +ih \to x + iy$, we get 
$$\lim_{h \to 0} \frac{f(z + h) - f(z)}{h} = \lim_{h \to 0} \frac{f(x + i(y + h)) - f(x + iy)}{ih} = \frac{f_y}{i} = -if_y = -i(u_y(x,y) + iv_y(x, y)) = v_y(x, y) -iu_y(x, y)$$
Equate components in $f_x = -if_y$, and it is proven that $u_x = v_y$ and $u_y = -v_x$. \\

\textbf{Proposition.} If the Cauchy Riemann Equations do not hold at $z_0$, then $f'(z_0)$ does not exist. \\
\textbf{Proposition.} If $f$ is holomorphic on a domain $\Omega$, an open and connected set in $\mathbb{C}$, then the Cauchy Riemann Equations hold at all points in $\Omega$. \\
\newline
\textbf{Example.} \\
If $f(x + iy) = x^2 + iy^2$, $u_x = 2x$, $v_x = 0$, $u_y = 0$, $v_y = 2y$. Then $2x = 2y \Rightarrow x = y$, which is a line. \\
The set of points on the line is not open in $\mathbb{C}$. Therefore, $f$ is nowhere holomorphic in $\mathbb{C}$. However, we will see that $f'(z)$ does exist on the line $y = x$. \\
\newline
\textbf{Sufficiency of the Cauchy Riemann Equations to $f'$.} \\
The Cauchy Riemann Equations do a great job showing $f'$ does not exist. But what about it being sufficient for $f'$? We claim that satisfying the Cauchy Riemann Equations at $z_0$ implies that $f'$ exists at $z_0$. \\
\textbf{Proof.}\\
$f$ is $\mathbb{C}$-differentiable at $z_0$ if and only if $u(x, y)$ and $v(x, y)$ have continuous partial derivatives that satisfy the Cauchy Riemann Equations at $z_0$. 
This requires us to treat $f(x + iy)$ as a function on $\mathbb{R} ^ 2$, or $f(z)$ induces a map on $\mathbb{R} ^ 2$. \\
Let $ h = \Delta x + i\Delta y$, 
$$ \frac{f(z + h) - f(z)}{h} = \frac{u(x + \Delta x, y + \Delta y) + iv(x + \Delta x, y + \Delta y)}{\Delta x + i \Delta y} - \frac{u(x, y) + iv(x, y)}{\Delta x + i \Delta y}$$
$$u(x + \Delta x, y + \Delta y) - u(x, y) = u(x + \Delta x, y + \Delta y) - u(x, y + \Delta y) + u(x, y + \Delta y) - u(x, y)$$ 
The function $u(\cdot, \cdot)$ is differentiable in $x$ and $y$, we can use the M.V.T (Mean Value Theorem) from $\mathbb{R}$ to rewrite our difference in $u$ by \\
$$u(x + \Delta x, y + \Delta y) - u(x, y + \Delta y) = \Delta xU_x(\underline x, y + \Delta y)$$
where $\underline x \in (x, x + \Delta x)$. \\
\newline 
If $u_x$ is continuous, $u_x(\underline x, y + \Delta y) \approx u_x(x, y) + \epsilon_1$, and as $\Delta y \to 0$ and $\underline x \to x$, by Taylor approximation and linear approximation on $u_x$, we have the error function $\epsilon_1 \to 0$. \\
Next $u(x, y + \Delta y) - u(x, y) = \Delta y u_y(x, \overline y)$ and $u_y(x, \overline y) \approx u_y(x, y) + \epsilon_2$. \\
Likewise, for the function $v(x, y)$, we get a $v_x$ and $v_y$ with error terms $\epsilon_3$ and $\epsilon_4$. s
$$\frac{f(z + h) - f(z)}{h} = \frac{\Delta x(u_x + \epsilon_1 + iv_x + i\epsilon_3) + \Delta y(u_y + \epsilon_2 + iv_y + i\epsilon_4)}{\Delta x + i \Delta y}$$
From the Cauchy Riemann Equations, we get $f_x = \frac{f_y}{i} \Rightarrow if_x = f_y \Rightarrow i(u_x + iv_x) = u_y + iv_y$. Substituting the terms, we have 
$$f'(z) = \frac{\Delta x(u_x + iv_x) + i \Delta y(u_x + iv_x)}{\Delta x + i \Delta y} + \frac{\lambda}{\Delta x + i\Delta y}$$
where $\lambda = \Delta x(\epsilon_1 + i\epsilon_2) + \Delta y(\epsilon_3 + i\epsilon_4)$. However, \\
$$\abs*{\frac{\lambda}{\Delta x+ i \Delta y}} \leqslant \abs*{\frac{\Delta x(\epsilon_1 + i\epsilon_2)}{\Delta x + i\Delta y}} + \abs*{\frac{\Delta y(\epsilon_3 + i\epsilon_4)}{\Delta x + i \Delta y}} \leqslant |\epsilon_1 + i\epsilon_2| + |\epsilon_3 + i\epsilon_4|$$ 
because $\abs*{\frac{\Delta x}{\Delta x + i\Delta y}} \leqslant 1$. \\
As $\Delta z \to 0$, $\abs*{\frac{\lambda}{\Delta x+ i \Delta y}} \to 0$, and thus $f'(z) = u_x + iv_x = f_x = \frac{f_y}{i}$. \\
Therefore, the Cauchy Riemann Equations are an easy way to show $f'(z)$ exists and they provide a set of partial differential equations that $f$ must satisfy. \\
\newline
\textbf{Example.} Let $f(z) = e^z = e^x(\cos(y) + i\sin(y))$ 
$$u = e^x\cos(y), v = e^x\sin(y)$$ 
$$u_x = e^x\cos(y), v_x = e^x\sin(y)$$
$$u_y = e^x\sin(y), v_x = e^x\cos(y)$$
Therefore, $f(z)$ is $\mathbb{C}$-differentiable on $\mathbb{C}$, $f$ is entire/meromorphic. $f'(z) = f_x = u_x + iv_x = f(z)$.

\subsection{Cauchy Riemann with Logarithm}
$$e^{\operatorname{Log(z)}} = z \Rightarrow \frac{d}{dz}e^{\operatorname{Log(z)}} = 1 \Rightarrow z\frac{d}{dz}\operatorname{Log}(z) = 1 \Rightarrow \frac{d}{dz}\operatorname{Log}(z) = \frac{1}{z}$$
We have a branch point in $\operatorname{Log}(z)$ where its derivative is undefined. Then $\operatorname{Log}(z)$ is $\mathbb{C}$-differentiable on $\mathbb{C} \backslash \{0\}$. This is true regardless of the branch cut on $\operatorname{Log}(z)$. 

\subsection{Lack of Complex Mean Value Theorem}
\textbf{Claim}: $\frac{f(z) - f(w)}{z - w} \neq f'(c)$ for some $c$ between $z$ and $w$. \\
\textbf{Proof}: Let $z = 1$, $w = 0$ and $f(t) = e^{i\pi t}$, then $f(1) - f(0) = e^{i\pi} - 1 = -2$. However, $|f'(t)| = \pi$ for all $t \in [0, 1]$. \\
\textbf{Follow-Up Question}: Does the lack of a Mean Value Theorem for $f'(z)$ suggest $f'(z) = 0$ not imply $f$ is constant?  \\
\textbf{Answer}: Suppose $f$ is $\mathbb{C}$-differentiable on $\Omega$ and one of the following holds, then $f$ is constant on $\Omega$. 
\[ \begin{cases} 
	f'(z) = 0 \\
	|f(z)| \mbox{ is constant} \\ 
	Re(f(z)) \mbox{ is constant} \\ 
	$f$ \mbox{'s conjugate is } \mathbb{C} \mbox{-differentiable on } \Omega
   \end{cases}
\]

\subsection{Wirtinger Equations}
There is another way to study the Cauchy Riemann Equations by introducing two operators: 
\begin{equation*}
	\frac{\partial f}{\partial z} = f_z \mbox{ and } \frac{\partial f}{\partial \conj{z}} = f_{\conj{z}}
\end{equation*}
$$ f(x, y) \equiv f(x + iy) = u(x, y) + iv(x, y)$$
$$f(x, y) = f(\Re(z), \Im(z)) = f(\frac{z + \conj{z}}{2}, \frac{z - \conj{z}}{2i})$$
From the chain rule, we get 
$$f_z = f_xx_z + f_yy_z = \frac{1}{2}f_x + \frac{1}{2i}f_y = \frac{1}{2}f_x - \frac{i}{2}f_y$$
$$f_{\conj{z}} = f_xx_{\conj{z}} + f_yy_{\conj{z}} = \frac{1}{2}f_x - \frac{1}{2i}f_y = \frac{1}{2}f_x + \frac{i}{2}f_y$$
where $f_x = u_x + iv_x $ and $f_y = u_y + iv_y$ \\
These are the Wirtinger Equations. 
\[ \begin{cases} 
	\frac{\partial}{\partial z}  = \frac{1}{2}(\frac{\partial}{\partial x} - i\frac{\partial}{\partial y}) \\ 
	\frac{\partial}{\partial \conj{z}}  = \frac{1}{2}(\frac{\partial}{\partial x} + i\frac{\partial}{\partial y})
   \end{cases}
\]
\newline
\textbf{Relationship with the Cauchy Riemann Equations.}\\
From the Cauchy Riemann Equations $if_x = f_y$ we get:
$$f_{\conj{z}} = \frac{1}{2}f_x + \frac{i}{2}f_y = \frac{1}{2}f_x - \frac{1}{2}f_x = 0$$
$$f_z = \frac{1}{2}f_x - \frac{i}{2}f_y = \frac{1}{2}f_x - \frac{i^2}{2}f_x = f_x = f'(z)$$
f is $\mathbb{C}$-differentiable at $z_0$ if and only if $f(x, y) = u(x, y) + iv(x, y)$ is $\mathbb{R}$-differentiable at $z_0$ and $f_{\conj{z}}(z_0) = 0$. Then $f'(z_0) = f_z(z_0)$. In other words, $f(z)$ does not depend on $\conj{z}$. 

\newpage
\section{Harmonic Functions}
\subsection{Laplacian}
Let $u: \mathbb{R}^2 \to \mathbb{R}$, then the Laplacian of $u$ is 
$$\Delta u = u_{xx} + u_{yy} = \nabla \cdot \nabla u$$
 where 
$\nabla = [\frac{\partial}{\partial x}, \frac{\partial}{\partial y}]^T$ is the divergence operator and $\nabla u = [u_x, u_y]^T$ is the gradient of $u$. \\
\subsection{Harmonic Functions}
If $\Delta u = 0$, then $u(x, y)$ satisfies Laplace's (partial differential) equation or $u$ is a harmonic function. \\
This means: 
\[ \begin{cases} 
	u \mbox{ is continuous} \\
	u \mbox{'s } \nth{1} \mbox{ and } \nth{2} \mbox{ order partial derivatives exist and are smooth. }
   \end{cases}
\]
\newline
\textbf{Proposition.} Suppose $f = u + iv$ is holomorphic on $\Omega$ where $u(x, y)$ and $v(x, y)$ have continuous $\nth{2}$ order partial derivatives, then $u$ and $v$ are harmonic and $v$ is the harmonic conjugate of $u$. \\
\textbf{Proof.} \\
By the Cauchy Riemann Equations, $u_x = v_y$ and $v_x = -u_y$, then $u_{xx} = v_{yx}$ and $v_{xy} = -u_{yy}$. By continuity of $v_{yx}$ and $v_{xy}$, $v_{yx} = v_{xy}$. This implies $u_{xx} = -u_{yy} \Rightarrow u_{xx} + u_{yy} = 0$ \\
Later on, we will find that the conditions on $\nth{2}$ order partial derivatives is implied by $f$ being holomorphic on $\Omega$, or $f''$ exists. \\
\newline
\textbf{Definition of Harmonic Conjugate.} \\
The harmonic conjugate to $u(x,y)$ is a function $v(x,y)$, such that $f(x,y) = u(x,y) + iv(x,y)$ is holomorphic. \\

\textbf{Example.} Show that $u(x, y) = x^3 -3xy^2 + y$ is a harmonic function. \\
$u_x = 3x^2 - 3y^2$, $u_y = -6xy + 1$ \\
$u_{xx} = 6x$, $u_{yy} = -6x$. Therefore, $u_{xx} + u_{yy} = 0$ \\

\textbf{Example.} Find the harmonic conjugate of $u(x, y) = x^3 - 3xy^2 + y$. \\
$u_x = 3x^2 - 3y^2 = v_y$ \\
$u_y = -6xy + 1 = -v_x$ \\
$\Rightarrow$ $v = 3x^2y - y^3 + C(x)$ or $v = 3x^2y - x + C(y)$ \\
Therefore, $v = 3x^2y - y^3 - x + C$ is $u$'s harmonic conjugate. \\
\newline 
\textbf{Proposition.} \\ 
If $u$ is harmonic on a domain $\Omega$, then $u_x$ is the real part of a holomorphic function on $\Omega$. If $\Omega$ is simply connected, unlike $\mathbb{C} \backslash \{0\}$, then $u$ is the real part of a holomorphic function on $\Omega$. \\
\textbf{Proof.} \\
Assume $u$ is harmonic and $\Omega$ is connected. If $f = u_x - iu_y$, then $f_y = if_x$. Hence, $f$ is differentiable on $\Omega$. \\
The simply connected statement requires future theorems to show $F'(z) = f(z)$ for some holomorphic antiderivative $F(z)$. 

\newpage
\section{Conformal Maps}
\textbf{Example.} Let $f(z) - (x + iy)^2 + 2(x +iy) = (x^2 + 2x - y^2) + i2(xy + y)$. When are the component functions, $u(x, y)$ and $v(x, y)$ constant? \\
When are the component functions , $u(x, y)$ and $v(x, y)$, constant? \\
The function $f(z) = e^z = e^x(\cos(y) + i\sin(y))$ maps the set $\Omega = \{z: \abs{\Im(z)} < \pi \}$ to circles of radius $r \in (-\infty, \infty)$, or all points in $\mathbb{C} \backslash \mathbb{R}^-$. This coincides with the branch cut of $\operatorname{Log}(z)$, or how we made $e^z$ invertible. \\

\subsection{Preservation of Angles}

We will now show $e^z$ preserves the angles between curves in $\Omega$. Let us first look at the following example. \\ 
Let $\gamma_1(t) = 2i\pi t - i\pi$, $\gamma_2(t) = t + i\frac{\pi}{4}$. $\gamma_1(0) = -i\pi$, $\gamma_1(1) = i\pi$. \\
The curves $\gamma_1$ and $\gamma_2$ intersect at an angle $\frac{\pi}{2}$. Also $f(\gamma_1)$ is a circle centered at 0 while $f(\gamma_2)$ is a line through $z = 0$. Their intersection in the $w$-plane is $\frac{\pi}{2}$ as well. \\
We will show why $f(z) = e^z$ does this by studying the angles between curves $\gamma_1$ and $\gamma_2$ and curves $\tau_1 = f(\gamma_1)$ and $\tau_2 = f(\gamma_2)$. 
If $\gamma(t)$ parameterizes a smooth curve in $\mathbb{C}$, then its tangent vector is $\gamma'(t)$. The angle between any two curves at $z_0$ is the angle between their tangent vectors at $z_0$. \\
Assume the curves intersect at $\gamma(r_0) = \gamma(s_0) = z_0$. \\
Let the angle of intersection, $\theta$, measured from $\gamma_1'$ to $\gamma_2'$ in the counter-clockwise direction. \\
Let the angle of intersection after transformation of $f$, $\varphi$, measured from $\tau_1'$ to $\tau_2'$ in the counter-clockwise direction. From past chapters, we know $\theta \approx \varphi$ if $f$ is holomorphic. Now, let us assume $f$ is only $\mathbb{R}$-differentiable and see how $f$ acts on the angle $\theta$. \\
Curve: $\gamma(t) = (x(t), y(t))$ \\
New curve, $f$ on $\gamma$: $\tau(t) = f(\gamma(t)) = u(\gamma(t)) + iv(\gamma(t)) = (\overline{\underline {X}}(t), \overline{\underline {Y}}(t))$ \\
New tangent vector: $\tau'(t) = \frac{t}{dt}\tau(t) = (\overline{\underline {X}}'(t), \overline{\underline {Y}}'(t))$, where we can invoke the chain rule: \\
$$\overline{\underline {X}}'(t) = u_x(\gamma(t))x'(t) + u_y(\gamma(t))y'(t)$$ 
$$\overline{\underline {Y}}'(t) = v_x(\gamma(t))x'(t) + v_y(\gamma(t))y'(t)$$
If we have $f = u(x, y) + iv(x, y)$ is $\mathbb{R}$-differentiable, then 
\begin{equation*}
J(f) = 
\begin{bmatrix}
 u_x & u_y \\
 v_x & v_y
\end{bmatrix}
\end{equation*}
is the Jacobian Matrix of $f$ and $\tau'(t) = \gamma'(t) \cdot J(f)^T$ \\
If $f$ is $\mathbb{C}$-differentiable and $\gamma(r_0) = z_0$ where $\gamma(t) = x(t) + iy(t)$, then $f'(z) = u_x + iv_x$ and $\gamma'(t) = x' + iy'$. \\
$$f'(z_0)\gamma'(r_0) = f'(\gamma(r_0))\gamma'(r_0) = (u_x + iv_x)(x' + iy') = (u_xx' - v_xy') + i(u_xy' + v_xx')$$
Applying the Cauchy Riemann Equations, 
\begin{equation*}
(u_xx' - v_xy') + i(u_xy' + v_xx') = (u_xx' + u_yy') + i(v_xx' + v_yy') = (u_xx' + u_yy', v_xx' + v_yy') \mbox{ in } \mathbb{R}^2 
\end{equation*}
$$ = \gamma'(r_0) \mbox{Jf}(z_0)^T = \tau'$$ 
By now, we have an understanding of how $f$ acts on tangent vectors when $f' \neq 0$, namely $\theta = \varphi$.

\subsection{Conformal Function}
\textbf{Conditions of Conformal Functions.}\\
We say $f$ is a conformal map at $z_0$ if the following hold: \\
\circled{\scriptsize1} $f$ is $\mathbb{R}^2$-differentiable at $z_0$ \\
\circled{\scriptsize2} $\abs{\mbox{Jf}} \neq 0$ \\
\circled{\scriptsize3} $f$ preserves the oriented angle $\theta$, between $\gamma_1$ and $\gamma_2$ and $\tau_1$ and $\tau_2$ at $z_0$ and $f(z_0)$. \\
\newline
\textbf{$e^z$ is conformal.}\\
Now let us take a closer look at $e^z$ and figure out why it is conformal. \\
\circled{\scriptsize1} holds apparently. \\
\circled{\scriptsize2} $f(z) = e^z = e^x\cos(y) + ie^x\sin(y)$, then 
\begin{equation*}
J(f(x, y)) = 
\begin{bmatrix}
 e^x\cos(y) & -e^x\sin(y) \\
 e^x\sin(y) & e^x\cos(y)
\end{bmatrix}
=e^x
\begin{bmatrix}
\cos(y) & -\sin(y) \\
\sin(y) & \cos(y)
\end{bmatrix}
\end{equation*}
$\Rightarrow \abs{J(f(x,y))} \neq 0$\\
\circled{\scriptsize3} Now we have shown in the previous part that Jf is the product of a dilation matrix, $e^xI$, and a rotation matrix, which means $f$ preserves the angles between $\gamma_1'$ and $\gamma_2'$. Their image under $f$: \\
$$\tau_1'(t) = \gamma_1'(t) \cdot J(f)^T$$
$$\tau_2'(t) = \gamma_2'(t) \cdot J(f)^T$$
In fact, $f$ preserving oriented angles implies Jf is a rotation $\otimes$ dilation matrix. \\
Hence, it is proven that $e^z$ is conformal.
\subsection{Conformal Map} 

\textbf{Definition.} If $f$ is conformal, infinitely differentiable, and one-to-one on a domain $\Omega$ to $V$, then $f$ is a conformal map from $\Omega$ to $V$. \\
For example, $e^z$ is conformal map from $\Omega = \{z:\abs{\Im(z)} < \pi \}$ to $V = \mathbb{C} \backslash \mathbb{R}^-$. \\
\textbf{Proposition.} If $f$ is complex differentiable and $f'(z_0) \neq 0$, it is a linear transform of a dilation by $\abs{f'(z_0)}$ and a rotation by $\operatorname{Arg}(f'(z_0))$. Hence, $f$ is conformal because \circled{3} is satisfied. \\
\textbf{Example.} $f(z) = z^2$ on $\Omega = \{z| 1< |z| < 3 \mbox{ and } \Im(z) > 0 \}$ is conformal. \\
\newline
\textbf{Inverse Function Theorem.} \\ 
If $f$ is a continuously differentiable function with nonzero derivative at the point $a$, then $f$ is invertible in a neighborhood of a, the inverse is continuously differentiable, and the derivative of the inverse function at $b=f(a)$ is the reciprocal of the derivative of $f$ at $a$: 
$$(f^{-1})'(b) = \frac{1}{f'(a)} = \frac{1}{f'(f^{-1}(b))}$$
\textbf{Proposition.} If $f$ is invertible at $z_0$ and conformal, then $f^{-1}$ is conformal at $f(z_0)$ by the inverse function theorem provided $f$ is continuously differentiable. \\
\newline 
From the proposition above, we know that $\operatorname{Log}(z)$ is conformal on $\mathbb{C} \backslash \mathbb{R}^-$.

\newpage
\section{Bilinear Transformations}
In complex analysis, the term linear transformation is used to describe affine transformations, $f(z) = az + b$. \\
\subsection{Definition of a Möbius transformation}
A bilinear/Möbius transformation is of the form 
$$\frac{az + b}{cz + d}$$
where $a, b, c, d \in \mathbb{C}$. \\
Now, $f(\infty) = \frac{a}{c}$ by L'Hospital argument and we say $f(-\frac{d}{c}) = \infty$. \\
If $ad - bc \neq 0$, then $f' \neq 0$ by the quotient rule and $f$ is not constant. The constants are not unique, as $f(z) = \frac{az + b}{cz + d} = \frac{(az+b)k}{(cz + d)k}$. Therefore, we only have 3 degrees of freedom. \\
\subsection{Brief Review of other Transformations}
We have already seen these functions of this form before $f$ is: \\
\circled{\scriptsize1} Composition of a finite number of 
\[ \begin{cases} 
	\mbox{Translations}, f(z) = z + k \\
	\mbox{Rotations}, f(z) = e^{i\theta}z \\
	\mbox{Dilations}, f(z) = kz, k \in \mathbb{R} \\
	\mbox{Inversions}, f(z) = \frac{1}{z}\\
   \end{cases}
\]
\circled{\scriptsize2} Conformal, $f$ is holomorphic away from $z = -\frac{d}{c}$, $f' \neq 0$, and f is one-to-one. \\
\circled{\scriptsize3} Maps circles/lines to either lines or circles, "lines are circles of $\infty$-radius in $\mathbb{C}$ or $\mathbb{C} \cup \{\infty\}$. \\
If the line or circle passes through $z = -\frac{d}{c}$, where $f$ is undefined, then it will be mapped to a line. Otherwise, it is mapped to a circle. \\
\circled{\scriptsize4} $f$ can be identified by 
\begin{equation*}
A = 
\begin{bmatrix}
 a & b \\
 c & d
\end{bmatrix}
\end{equation*}
Then,
\[ \begin{cases} 
	f \circ f \equiv A \circ A = A^2 \\
	f^{-1} \equiv A^{-1} \\
	f \circ g \equiv AB \\
   \end{cases}
\]
and so there is a group homomorphism with Möbius transforms and invertible matrices in $\mathbb{C}^{2 \times 2}$. \\
$f(z) = (3 + 2i)z - i^3 = \frac{(3 + 2i)z - i^3}{\delta z + 1}$ \\
\circled{\scriptsize5} We can conformally map 3 points in $\mathbb{C} \cup \{\infty\}$ to any 3 points in $\mathbb{C} \cup \{\infty\}$. \\
This type of argument is similar to showing norms are equivalent in $\mathbb{R}^n$ or uniqueness of power series expansions. \\
\subsection{Möbius transforming of a function}
Given any 3 points $z_0$, $z_1$, $z_2$ $\in \mathbb{C}$, we can create a Möbius transformation $T$ such that  

\begin{alignat*}{2}
    & \begin{aligned} & \begin{cases}
  T(z_0) = 0\\
  T(z_1) = 1\\
  T(z_2) = \infty \\
  \end{cases}\\
  \MoveEqLeft[-1]\text{}
  \end{aligned}
    & \hskip 6em &
  \begin{aligned}
  & \begin{cases}
  T^{-1}(0) = z_0\\
  T^{-1}(1) = z_1\\
  T^{-1}(\infty) = z_2\\
  \end{cases} \\[0ex]
  \MoveEqLeft[-1]\text{}
  \end{aligned}
\end{alignat*}

Then 
$$T(z) = (z, z_0, z_1, z_2) = \frac{(z -z_0)(z_1 - z_2)}{(z - z_2)(z_1 - z_0)}$$
is called the cross-ratio of $z$, $z_0$, $z_1$, and $z_2$. \\
There are some special cases: 
\[ \begin{cases} 
	(z, \infty, z_1, z_2) = \frac{z_1 - z_2}{z - z_2} \\
	(z, z_0, \infty, z_2) = \frac{z - z_0}{z - z_2} \\
	(z, z_0, z_1, \infty) = \frac{z - z_0}{z_1 - z_0}\\
   \end{cases}
\]
Given 3 more points $w_0, w_1, w_2 \in \mathbb{C}$, we get $S(w) = (w, w_0, w_1, w_2)$. Then, we can construct the function map: 
$$ z_0 \overset{T}{\longrightarrow} 0 \overset{S}{\longleftarrow} w_0 $$
$$ z_1 \longrightarrow 0 \longleftarrow w_1$$
$$ z_2 \longrightarrow 0 \longleftarrow w_2$$

%$$A\xleftarrow{n+\mu-1}B \xrightarrow[T]{n\pm i-1}C$$
%$\overset{x}{\longrightarrow} ( $ \overset{x}{\longrightarrow}$).$
Then $w = f(z) = S^{-1} \circ T(z)$ maps $z_0 \to w_0$, $z_1 \to w_1$, and $z_2 \to w_2$. \\
The points must be distinct, because $f$ is one-to-one. \\
To find $f$ from above, we solve $(z, z_0, z_1, z_2) = (w, w_0, w_1, w_2)$ for $w$. \\
\newline
\textbf{Example.} \\
Suppose 
$$z_0 = 1 \to i = w_0$$
$$z_1 = -1 \to 1 = w_1$$
$$z_2 = 1 \to -1 = w_2$$
then 
$$\frac{(w - i)(1 - (-1))}{(w - (-1))(1-i)} = \frac{2(w - i)}{(w + 1)(1 - i)} = \frac{(z - i)(-1-1)}{(z - 1)(-1 - i)} = \frac{-2(z - i)}{(z-1)(-1-i)} = \frac{2(z-i)}{(z-1)(1+i)}$$
$$ \Rightarrow \frac{w - i}{(w +1)(1 -i )} = \frac{z - i}{(z - 1)(1 + i)}$$
$$ \Rightarrow (w - i)(z - 1)(1 + i) = (w + 1)(1 - i)(z - i)$$
$$ \Rightarrow w = -\frac{1}{z}$$
It is much work to find a simple function. Mapping a set of three points to another set of three points is time-consuming. If we study Möbius transforms as conformal mappings, then we can introduce another way to move the three points around. \\

\textbf{Example.} Find a Möbius transform to map the unit disk $|z| < 1$ to $\Im(z) > 0$. \\
If we pick where 3 points go, say on the boundary of $\Omega$, to the boundary of $\Im(z) > 0$, which is the $\mathbb{R}$-axis, then we can construct $f(z)$. We also need one point to get mapped to $\infty$, but we have $\infty$-many choices for the three points. \\
We want $f(-1) = 0$, $f(-1) = 1$, and $f(1) = \infty$, that will be 
$$f(z) = (z, -1, -i, 1) = \frac{(z - z_0)(z_1 - z_2)}{(z - z_2)(z_1 - z_0)} = \frac{(z + 1)(-i - 1)}{(z - 1)(-i + 1)} = -i\frac{z + 1}{z - 1}$$
Our path's direction around the circle has to be preserved because $f$ is conformal. \\
This means points on the interior of the circle will get mapped to the upper half of $\mathbb{C}$. \\
To check, we see that $f(0) = i, \Im(f(0)) > 0$. Therefore, the function we found out is correct. \\

\textbf{Example.} Find a Möbius transform to map the unit disk $|z| < 1$ to $\Im(z) > 0$ and $\Re(z) > 0$. \\
Using the conclusion from the example above, the new desired function is simply 
$g(z) = (f(z))^{\frac{1}{2}}$.

\newpage
\section{Contour Integral in $\mathbb{C}$}
Let $\gamma(t) = x(t) + iy(t)$ be a curve in $\mathbb{C}$ where $\gamma(a) = z_0$ and $\gamma(b) = z_1$. \\
Let $C$ be the graph of $\gamma(t)$, $C = \{z | z = \gamma(t) \mbox{ for some } t \in [a, b] \}$. 
\subsection{Piecewise Differentiable, Smooth, Simple, Closed curves}
\textbf{Piecewise Differentiable curves.} \\
The curve determined by $\gamma$, its graph $C$, is considered piecewise differentiable if 
\begin{enumerate}
\item $x$ and $y$ are continuous on [a, b]
\item $x'$ and $y'$ are continuous on a partition of $[a, b]$, $[x_0, x_1] \cup [x_1, x_2] \cup [x_2, x_3] \cup \cdots \cup [x_{n - 1}, x_n]$
\end{enumerate} 

\textbf{Smooth curves.} \\
If $\gamma' \neq 0$ for only finitely many points, then the curve is considered smooth. \\

\textbf{Simple curves.} \\
A curve is simple if it does not intersect itself, i.e. $\gamma(t) = \gamma(s)$ if and only if $s = t$. \\

\textbf{Closed curves.} \\
$C$ is a closed curve if it starts and stops at the same point, i.e. $\gamma(a) = \gamma(b), t \in [a, b]$ 

\subsection{Interior and Exterior of curves} 
A closed and simple curve keeps the interior of the set on its left side and its exterior to the right. \\
This means we traverse circles counter-clockwise to describe their interior correctly. \\
\newline
\textbf{Jordan Curve Theorem.} \\
A closed and simple curve partitions $\mathbb{C}$ into two regions, one of them bounded, defined as the interior of the curve. 

\subsection{Smoothly Equivalent}
The parameter $t$ provides an orientation or direction to $C$. \\
Let 
\[ \begin{cases} 
	C_1: \gamma_1(t), t \in [a, b] \\
	C_2: \gamma_2(t), t \in [c, d]\\
   \end{cases}
\]
We say $C_1$ and $C_2$ are smoothly equivalent if there exists a one-to-one, continuous derivative mapping $\lambda(t)$, 
$$\lambda(t): [c, d] \to [a, b]$$ 
$$\lambda(c) = a$$
$$\lambda(d) = b$$ 
$$\lambda'(t) > 0$$
\begin{equation*}
\mbox{where } \gamma_1(\lambda(t)) = \gamma_2(t)
\end{equation*}
\textbf{Example.}
\[ \begin{cases} 
	C_1: \gamma_1(t) = \cos(t) + i\sin(t), t \in [0, 2\pi] \\
	C_2: \gamma_2(t) = \cos(2t) + i\sin(2t),  t \in [0, \pi]\\
   \end{cases}
\]
Here $C_1$ and $C_2$ are smoothly equivalent, since we can let $\lambda(t) = 2t$. \\
Both parametrize the unit circle, preserve the orientation, and pass through each point the same number of times. \\
Let us look at other two curves: 
\[ \begin{cases} 
	\gamma_3(t) = \cos(4t) + i\sin(4t), t \in [0, \pi] \\
	\gamma_4(t) = \cos(t) + i\sin(-t),  t \in [0, 2\pi]\\
   \end{cases}
\]
$\gamma_3$ traverses the circle multiple times and $\gamma_4$ has the opposite orientation of $\gamma_1$ and $\gamma_2$. \\
\newline
Let $-C$ be the curve $C$ but with a reversed orientation, $\gamma_{R}(t) = \gamma(b + a -t)$. 
\[ \begin{cases} 
	C_1: \gamma_1(t) = \cos(t) + i\sin(t), t \in [0, 2\pi] \\
	-C_1: \gamma_4(t) = \cos(t) + i\sin(-t),  t \in [0, 2\pi]\\
   \end{cases}
\]
We want to integrate $f(z)$ over curves $C$ in $\mathbb{C}$. These will factor into the computation: 
\[ \begin{cases} 
	\mbox{Orientation of } C\\
	\mbox{Number of times } C \mbox{ traverses itself} \\
	C \mbox{ is closed} \\
	f \mbox{ is holomorphic on the interior of } C \mbox{ and } C
   \end{cases}
\]
\subsection{Line Integral}
The line integral of $f$ over $C$ is given by 
$$\int_C f(z) \,dz = \int_C u(z) + iv(z) \,dz = \int_C u(z) + iv(z) \,{(dx + idy)} = \int_{a}^{b} f(\gamma(t))\gamma'(t) \,dt$$
If $C$ is a closed curve, $\gamma(a) = \gamma(b)$, then we can use a closed loop in our $\int$ symbol, $\oint_C f(z) \,dz$ \\
The term $\gamma'(t)dt$ controls for how fast we traverse the curve. The integral is independent of our choice of smoothly equivalent curve, $C_1$ or $C_2$. \\
\newline
\textbf{Proposition.} Let $C_1$ and $C_2$ be smoothly equivalent, \\
$$\int_{C_1} f(z) \,dz = \int_{C_2} f(z) \,dz$$
\textbf{Proof.} Let $\gamma_1(\lambda(t)) = \gamma_2(t)$ and apply change of variables. \\
Because $u = \lambda(t), u(c) = a, u(d) = b$ and $\gamma_1'(\lambda(t))\lambda'(t) = \gamma_2'(t)$, 
$$\int_{c}^{d} f(\gamma_2(t))\gamma_2'(t)\,dt = \int_{c}^{d}f(\gamma_1(\lambda(t)))\gamma_1'(\lambda(t))\lambda'(t)\,dt = \int_{a}^{b}f(\gamma_1(u))\gamma_1'(u)\,du$$
\textbf{Proposition.} 
$$-\int_{C} f(z) \,dz = \int_{-C} f(z) \,dz$$
\textbf{Proof.}
$$\int_{-C} f(z) \,dz = \int_{b}^{a}f(\gamma_R(t))\gamma_R'(t) \,dt = \int_{b}^{a}f(\gamma(t))\gamma'(t) \,dt$$
\newline
\textbf{Proposition.} Linearity holds: 
$$\int_C \alpha f(z) + g(z) \,dz = \alpha \int_C f(z) \,dz + \int_C g(z) \,dz$$

\textbf{Example.} Find $\oint_{|z| = 1} \frac{1}{z} \,dz = \int_{0}^{2\pi} f(\gamma(t))\gamma'(t) \,dt$ \\
We parametrize the curve as follows: \\
$C$: $\gamma(t) = \cos(t) + i\sin(t)$, $t \in [0, 2\pi]$,  $\gamma'(t) = -\sin(t) + i\cos(t)$ \\
$$\frac{1}{z} = \frac{x}{x^2 + y^2} - i\frac{y}{x^2 + y^2}$$
$$\int_0^{2\pi}f(\gamma(t))\gamma'(t) \,dt = \int_0^{2\pi}(\cos(t) - i\sin(t))(-\sin(t) + i\cos(t)) \,dt = \int_0^{2\pi}i \,dt = 2\pi i$$
There is another way to parametrize the curve: \\
$\gamma(t) = re^{it}$, $\gamma'(t) = ire^{it}$
$$\oint_{|z| = r} \frac{1}{z} \,dz = \int_0^{2\pi} f(re^{it})ire^{it} \,dt = i\int_0^{2\pi}e^{-it}e^{it} \,dt = \int_0^{2\pi}i \,dt = 2\pi i$$
From this example, we have the following famous result. \\
\newline
\textbf{Proposition.} The following holds with proof shown above.
\begin{equation*} 
\oint_{|z| = r} z^k \,dz = 0 \mbox{ for all } k \in \mathbb{Z}, k \neq -1, r > 0. 
\end{equation*}
\begin{equation*}
\oint_{|z| = r} \frac{1}{z} \,dz = \int_0^{2\pi} i \,dt = 2\pi i \mbox{ for r} > 0 
\end{equation*}
\newline
\textbf{ML Estimate.} \\
Let $f$ be a complex-valued, continuous function and $|f(z)| < M$ on $C$, a curve of length $L$, then 
$$|\int_Cf(z) \,dz| \leqslant M \cdot L$$
\textbf{Proof.}
$$ \abs*{\int_a^b f(\gamma(t))\gamma'(t) \,dt} \leqslant \int_a^b|f(\gamma(t))\gamma'(t)| \,dt \leqslant M\int_a^b|\gamma_1'(t)| \,dt = M \cdot L$$
where $\int_a^b|\gamma_1'(t)| \,dt$ is the arclength of $\gamma$. \\
We do not have to worry about the orientation of $C$, 
$$ \abs*{\int_{-C}f(z) \,dz} \leqslant M \cdot L$$
\newpage

\section{Cauchy's Closed Curve Theorem and the Fundamental Theorem of Calculus}
We now study contour integrals of holomorphic functions over closed curves. 
\subsection{Cauchy's Closed Curve Theorem}
\textbf{Green's Theorem.} \\
If $P(x, y)$ and $Q(x, y)$ have continuous first order derivatives on $\Omega$ and its boundary $\partial \Omega$, $\Omega$ is a domain, 
$$\oint_{\partial \Omega} P \,dx + Q \,dy = \int_\Omega Q_x - P_y \,dxdy$$
\newline
\textbf{Cauchy-Goursat Theorem.} \\
If $f(z)$ is holomorphic on $\Omega$ and $C$ is any closed curve in $\Omega$, where $\Omega$ is simply connected. Then
$$\oint_{C} f(z) \, dz = 0$$
\textbf{Proof.}
$$\oint_{\partial \Omega} f(z) \, dz = \oint_{\partial \Omega} u + iv \,{(dx + idy)} = \int_{\Omega}{if_x - f_y} \, dA$$
and the Cauchy Riemann Equations imply 
$$if_x = f_y$$
Therefore, 
$$\oint_{\partial \Omega} f(z) \, dz = 0$$
\newline
\textbf{Corollary of Cauchy's Theorem.} \\
Let $C_1$ and $C_2$ be paths from $z_0$ to $z_1$. Then $C_1 \cup -C_2$ is a closed loop in $\mathbb{C}$. $C_1$ and $C_2$ are in a simply connected subset of where $f$ is holomorphic. Then by Cauchy-Goursat Theorem, 
$$ 0 = \oint_{C_1 \cup -C_2}f(z) \,dz  = \int_{C_1}f(z) \,dz + \int_{-C_2}f(z) \,dz = \int_{C_1}f(z) \,dz - \int_{C_2}f(z) \,dz$$ 
Hence, 
$$\int_{C_1}f(z) \,dz = \int_{C_2}f(z) \,dz$$
\newline
This means it does not matter how you go from $z_0$ to $z_1$ as long as $f$ is $\mathbb{C}$ -differentiable along the way. This means we can use 
$$\int_{z_0}^{z_1}f(z)\,dz$$ 
to denote the integral from $z_0$ to $z_1$. But we still have to watch out for branch cuts. \\
Even non-simple curves are allowed, because Cauchy-Goursat Theorem tells us the integral over the loops in the non-simple curves are 0. \\
\newline
\textbf{Path Independence.} We can characterize Cauchy-Goursat Theorem with path independence. We have path independence of $\int_{z_0}^{z_1}f(z) \,dz$ in $\Omega$ if and only if 
\begin{equation*}
\oint_C f(z) \,dz = 0 \mbox{ for all closed curves } C \mbox{ in } \Omega
\end{equation*}
These conditions tell us holomorphic functions must have antiderivatives. 
\subsection{Fundamental Theorem of Calculus (F.T.C)} 
Path independence leads to introduction of Fundamental Theorem of Calculus. \\
\newline
\textbf{Fundamental Theorem of Calculus (Part One).} \\
Assume $f$ is holomorphic on $\Omega$, open and simply connected in $\mathbb{C}$. The function 
$$F(z) = \int_a^z f(u) \,du$$
is holomorphic on $\Omega$ and $F'(z) = f(z)$. \\
\textbf{Proof.} This is completely analogous to the proof on $\mathbb{R}$, which has two steps.
\begin{enumerate}
\item Compute difference quotient 
\item Bound integrand by invoking continuity of $f$ 
\end{enumerate}
\circled{\scriptsize1}
\begin{equation*}
\frac{F(z+h) - F(z)}{h} = \frac{1}{h}\int_z^{z+h} f(u) \,du 
\end{equation*}
$$\frac{1}{h}\int_z^{z+h}\,du = \frac{h}{h} = 1$$
implies
$$f(z) = f(z)\frac{1}{h}\int_z^{z+h}\,du = \frac{1}{h}\int_z^{z+h}f(z) \,du$$
Then 
$$\frac{F(z+h) - F(z)}{h} - f(z) = \frac{1}{h}\int_z^{z+h}f(u) - f(z) \,du$$
\circled{\scriptsize2} Since integrating over any closed curve $C$ gives 
$$\oint_C f(z) \,dz = 0$$
We can pick how we go from $z$ to $z +h$, say along a line from $z$ to $z + h$. We can do this for $h$ sufficiently small so that our path is in $\Omega$. This path is selected to make our upcoming estimate on $f(u) - f(z)$ easier to work with. \\
Along the path we can have $|u - z| < \delta$, which implies $|f(z) - f(u)| < \epsilon$ by continuity of $f$ where $u$ is on the path. From ML-Estimate, we get 
$$\abs*{\frac{F(z+h) - F(z)}{h} - f(z)} \leqslant \frac{1}{|h|}\int_z^{z+h}|f(u) - f(z)|\,du < \frac{1}{|h|}|h|\epsilon = \epsilon$$
Therefore, it is proven that $F'(z) = f(z)$. \\
\newline
Let us take a closer look at our proof above. We needed $f$ to be continuous so that we could invoke ML-Estimate. We also needed to pick a nice path from $z$ to $z + h$. But we did not need $f$ to be holomorphic. This leads to the next Theorem. \\
\newline
\newline
\textbf{Morera's Theorem.} Let $f$ be a continuous, complex-valued function defined on an open set D in the complex plane. If for all closed paths $C$ in $\Omega$, $\oint_C f(z) \,dz = 0$, then $F(z)$ is holomorphic on $\Omega$, $F'(z) = f(z)$ and $f$ is holomorphic. \\
Later we will show that $f'(z)$ is holomorphic when $f$ is holomorphic. \\
\newline
Let $f_n \to f$ uniformly on every compact subset of $\Omega$. We say $f_n$ converges on compacta to $f$. If $f_n$ is holomorphic on $\Omega$ for all $n$ and $C$ is any closed curve in $\Omega$, 
\begin{equation*}
\oint_C f_n(z) \,dz = 0 \mbox{ for all } n
\end{equation*}
$$\Rightarrow \lim_{n\to \infty}{\oint_Cf_n(z) \, dz} = 0$$
Since the convergence is uniform, $f$ is continuous and 
$$ \lim_{n\to \infty}{\oint_C f_n(z) \,dz} = \oint_C f(z) \, dz = 0$$
Hence $f$ is holomorphic on $\Omega$ by Morera's Theorem. \\
\newline 
\textbf{Example.} Let $f(z) = \int_0^{\infty} \frac{e^{zt}}{1 + t} \, dt$, $\Re(z) < 0$. \\
Because $\Re(z) < 0$, 
$$\int_0^{\infty} \frac{|e^{zt}|}{1 + t} \,dt \leqslant \int_0^{\infty} e^{\Re(z)t} \,dt = \Eval{\frac{e^{\Re(z)t}}{\Re(z)}}{\infty}{t = 0} = \frac{-1}{\Re(z)}$$
And so we have an absolute convergent integral that is bounded. 
$$|f(z)| \leqslant \frac{1}{|\Re(z)|}$$
Then, since the integral converges absolutely,
$$\oint_C f(z) \,dz = \oint_C \int_0^{\infty} \frac{e^{zt}}{1 + t}\,dtdz = \int_0^{\infty} \frac{1}{1+t} \oint_C e^{zt} \,dzdt = \int_0^{\infty} \frac{1}{1+t}0 \,dt = 0$$
by Cauchy-Goursat Theorem since $e^zt$ is entire for $t$ fixed. Therefore, $f(z)$ is holomorphic on $\Re(z) < 0$. \\
\newline 
We can generalize our result to handle other types of integral transforms $F(z) = \int_0^{\infty} K(t, z, f(t)) \,dt$. \\
We also have an easy way to evaluate the line integrals. \\
\newline
\textbf{Fundamental Theorem of Calculus (Part Two).} \\
Let $F(z)$ be $\mathbb{C}$-differentiable on a smooth curve $C$ where $F'(z) = f(z)$, then
$$\int_a^b f(z) \,dz = F(z)\Big|_{z = a}^{b}$$
\textbf{Proof.} Let $a, b, c \in \Omega$, where $c$ is in the middle of $a$ and $b$. Then, \\
$$F(b) - F(a) = \int_c^b f(u) \,du - \int_c^a f(u) \,du = \int_c^b f(u) \,du + \int_a^c f(u) \,du = \int_a^b f(z) \,dz$$
The result can also be shown with variable substitution. 
$$ \int_b^a f(z) \,dz = \int_0^1 f(\gamma(t))\gamma'(t) \,dt = F(\gamma(t))\Big|_{t = 0}^{1} = F(b) - F(a)$$
\newline
\textbf{Proposition.} If $f'(z) = 0$ on $\Omega$, then $f$ is constant on $\Omega$. i.e. $0 = u_x = v_y$ and $u_y = -v_x = 0$. \\
\textbf{Proof.} Integrate $f'$ along a staircase path in $\Omega$, 
$$f(z) = f(a) + \int_a^z 0 \, du = f(a)$$
Or we can use the Cauchy Riemann Equations.\\
\newline
Let $F'(z) = G'(z) = f(z)$, then $H(z) = F(z) - G(z)$ satisfies $H'(z) = F'(z) - G'(z) = 0$. This means $H$ is constant. This leads to the theorem below.  \\
\newline
\textbf{Theorem.} Any antiderivative of holomorphic $f(z)$ is of the form 
$$F(z) = \int_a^z f(u) \, du + C, C \in \mathbb{C}$$ 
\newline
\textbf{Proposition.} Let $\Omega$ be simply-connected, $0 \notin \Omega$. Fix $z_0 \in \Omega$ and pick the value of $\operatorname{log}z_0$. Then 
$$F(z) = \int_{z_0}^z \frac{1}{w} \,dw + \operatorname{log} z_0$$
is a branch of $\operatorname{Log} z $ in $\Omega$.\\
\textbf{Proof.} $F$ is well-defined since $\frac{1}{w}$ is holomorphic in $\Omega$, we have path independence from $z_0$ to $z$, $f' = \frac{1}{z}$. 
We still need $e^{F(z)} = z$.
Define 
$$g(z) = ze^{-F(z)}$$
$$g'(z) = e^{-F(z)}(1 - zF'(z)) = e^{-F(z)}(1 - \frac{z}{z}) = 0$$
Therefore, $g(z)$ is constant. 
$$g(z) = g(z_0) = z_0e^{-F(z_0)} = z_0 e^{-\operatorname{log}(z_0)} = 1$$
Therefore, 
$$e^{F(z)} = z$$

\newpage
\section{Cauchy's Integral Formula}
\textbf{First Extension of Cauchy-Goursat Theorem.} \\
Let $\Omega$ be the domain between two simple closed curves $C_1$ and $C_2$, each oriented counter-clockwise. If $f$ is holomorphic on $\Omega$, then 
$$ \oint_{C_1}f(z) \,dz = \oint_{C_2}f(z) \,dz $$
or
$$\oint_{C_1 \cup -C_2}f(z) \, dz$$
\newline
\newline
\textbf{Second Extension of Cauchy-Goursat Theorem. } \\
Let $g(z)$ be holomorphic on $\Omega \backslash \{z_0\}$, $g(z)$ is continuous on $\Omega$, $g(z_0)$ is finite. Then
$$\oint_C g(z) \,dz = 0$$
for any closed curve $C$ in $\Omega$ and $g(z)$ is holomorphic in $\Omega$. \\
\textbf{Proof.} By First Extension of Cauchy-Goursat Theorem 
$$\oint_C g(z) \,dz = \oint_{|z - z_0| = \epsilon} g(z) \, dz$$
and 
$$\oint_{|z - z_0| = \epsilon} |g(z)| \, dz \leqslant M_{\epsilon}\cdot 2\pi \cdot \epsilon$$
Since $g$ is continuous, $M_\epsilon$ is bounded and as $\epsilon \to 0$, we have 
$$\oint_C g(z) \,dz = 0$$
Therefore, $g$ is holomorphic on $\Omega$, $g'(z_0)$ exists by Morera's Theorem. \\
\newline
The second extension of Cauchy-Goursat Theorem gives us Cauchy Integral Formula. \\
Assume $f$ is holomorphic on $\Omega$, $0 \in \Omega$. Let 
$$g(z) = \frac{f(z) - f(0)}{z - 0} = \frac{f(z) - f(0)}{z}$$ 
as $z \to 0$, $g(0) = f'(0)$. Our function $g(z)$ is holomorphic on $\Omega \backslash \{z_0\}$ and continuous on $\Omega$, so the second extension of Cauchy-Goursat Theorem holds. \\
Let $C$ be a curve in $\Omega$, by our last result 
$$\oint_C g(z) \, dz = \oint_C \frac{f(z) - f(0)}{z} \, dz = 0$$
$$\Rightarrow \oint_C \frac{f(z)}{z} \, dz = \oint_C \frac{f(0)}{z} \, dz = f(0)\oint_C \frac{dz}{z} = 2\pi if(0)$$
We showed before that 
$$ \oint_C \frac{dz}{z} = 2\pi i$$
Therefore, 
$$f(0) = \frac{1}{2\pi i} \oint_C \frac{f(z)}{z} \,dz$$
This means that the value of $f$ at $z = 0$ only depends on the value of $\frac{f(z)}{z}$ over circles centered at 0. This result can be further generalized to Cauchy Integral Formula. \\
\newline
\textbf{Cauchy's Integral Formula. (C.I.F)} \\
Let $C$ be a smooth closed curve with interior domain $\Omega$. If $f$ is holomorphic on $\Omega$ and continuous on $\Omega$'s closure, 
$$f(z) = \frac{1}{2\pi i}\oint_C \frac{f(w)}{w - z} \, dw, z \in \Omega$$
\newline 
\textbf{Example.}
$$ \oint_{|z| = 6} \frac{e^{z^2}}{z - 4} = 2\pi if(4) = 2\pi i e^{16}, f(z) = e^{z^2}$$
However, 
\begin{equation*}
\oint_{|z| = 3} \frac{e^{z^2}}{z - 4} = 0 \mbox{ by Cauchy-Goursat Theorem }
\end{equation*}
\newline 
\newline
\textbf{Extension of Cauchy's Integral Formula.} \\
Cauchy's Integral Formula can be naturally expanded to $f'(z)$, 
$$ f'(z) = \frac{1}{2\pi i}\oint_C \frac{f'(w)}{w - z} \, dw = \frac{1}{2\pi i}\oint_C \frac{f(w)}{(w - z)^2} \, dw, z \in \Omega$$
Therefore, the result can be generalized to 
$$f^{(k)}(z) = \frac{k!}{2\pi i}\oint_C \frac{f(w)}{(w - z)^{k + 1}} \, dw, z \in \Omega$$
provided we can justify 
$$\frac{d}{dz} \oint_C \,dw = \oint_C \frac{d}{dz} \, dw$$
\textbf{Proof.} The proof of this depends on $\frac{1}{w - z}$ being differentiable in $w$ along any curve that does not pass through $z$. \\
Let $f$ be holomorphic on $\Omega$, then for any circle $C$ in $\Omega$ about $z_0$, we have a formula for $f^{(k)}(z_0)$ in terms of integrating over $C$. Hence $f^{(k)}$ exists on $\Omega$, $f^{(k + 1)}$ does 
\begin{align*}
&\Rightarrow f^{(k)}(z) \mbox{ is holomorphic on } \Omega \mbox{ for all } k \\ 
&\Rightarrow f \mbox{ is infinitely differentiable on } \Omega \\ 
&\Rightarrow f = u + iv \mbox{ holomorphic } \\
&\Rightarrow \mbox{All partials of } u \mbox{ and } v \mbox{ exist and are continuous}, u_{xy} = u_{yx} 
\end{align*}
\newline
\textbf{Example.} Given an integral in the form of C.I.F, we can take multiple approaches. Evaluate 
$$\oint_{|z| = 9} \frac{\sin(z)}{z^2} \,dz$$
The first approach is to apply C.I.F directly. Here, $f(z) = \sin(z)$, $f'(z) = \cos(z)$. 
$$ \oint_{|z| = 9} \frac{\sin(z)}{z^2} \,dz = 2\pi i f'(0) = 2\pi i \cos(0) = 2\pi i$$
The second approach is to apply C.I.F indirectly. The function
\[ 
g(z) = 
\begin{cases} 
	\frac{\sin(z) - \sin(0)}{z - 0}, z \neq 0 \\
	1, z = 0\\
   \end{cases}
\]
is entire by the second extension of Cauchy-Goursat Theorem. 
$$\oint_{|z| = 9} \frac{\frac{\sin(z)}{z}}{z} \,dz = 2\pi i g(0) = 2\pi i$$
\newline
\textbf{Gauss's Mean Value Theorem.} \\
Let $f(z)$ be an analytic function in $|z - z_0| < r$. Then 
$$f(z) = \frac{1}{2\pi}\int_0^{2\pi} f(z + re^{i\theta}) \,d\theta$$ 
\textbf{Proof.} Let $\Omega = \{z|\abs{z - z_0} < r\}$ \\
Let 
$$\gamma(t) = z_0 + re^{2\pi i t}$$
Then, 
\begin{align*}
f(z_0) &= \frac{1}{2\pi i} \oint_{\partial \Omega} \frac{f(z)}{z - z_0} \,dz \\
&= \frac{1}{2\pi i }\int_0^1 \frac{f(\gamma(t))}{\gamma(t) - z_0} \gamma'(t) \, dt \\
&= \int_0^1 f(z_0 + re^{2\pi it}) \,dt \\ 
&= \frac{1}{2\pi}\int_0^{2\pi} f(z_0 + re^{it}) \,dt
\end{align*}

\newpage
\section{Growth Conditions of Holomorphic Functions}
\subsection{Maximum/Minimum Modulus Principles}
If $f$ has an antiderivative on a domain $\Omega$, then $f$ is holomorphic on $\Omega$. However, its converse is not true. \\
If $f$ is holomorphic on $\Omega$, $f$ does not necessarily have an antiderivative. Here is a counter-example.\\
\newline
\textbf{Example.} Let $f(z) = \frac{1}{z}, \Omega = \mathbb{C} \backslash \{0\}$. \\
$f(z)$ does not have an antiderivative on $\Omega$ because 
$$\oint_{|z| = r}\frac{1}{z} \,dz = 2\pi i \neq 0$$
\newline
The problem above is with $\Omega$, which is not simply-connected. \\
\newline
\textbf{Proposition.} If $\Omega$ is simply-connected and $f$ is holomorphic on $\Omega$, then $f$ has an antiderivative by F.T.C, a corollary of Cauchy-Goursat Theorem. \\
\newline
\textbf{Maximum Modulus Principle.} \\
Let $f$ be holomorphic on a bounded domain $\Omega$. If $|f|$ has a local maximum at $z_0$, or in other words, 
$$|f(z_0)| > |f(z)|$$
for all $z$ in the neighborhood of $z_0$, 
then $f$ is constant near $z_0$. \\
In fact this also implies $f$ is constant in $\Omega$ with some additional theorems about uniqueness of holomorphic functions. \\
If $f$ is continuous on $\partial \Omega$, then either $f$ is constant or the absolute maximum of $|f|$ occurs only on the boundary of $\Omega$, $\partial \Omega$. \\
\textbf{Proof.} Let $f$ be holomorphic on a domain $\Omega$ and suppose the function $|f|$ takes on a local maximum value at $z_0$. Near $z_0$, 
$$|z -z_0| < r \Rightarrow |f(z)| \leqslant |f(z_0)|$$
Then by Gauss's Mean Value Theorem, 
$$|f(z_0)| = \abs*{\frac{1}{2\pi} \int_0^{2\pi}f(z_0 + re^{it}) \,dt} \leqslant \frac{1}{2\pi}\int_0^{2\pi} |f(z_0 + re^{it})| \,dt \leqslant \frac{1}{2\pi}\int_0^{2\pi}|f(z_0)| \,dt = |f(z_0)|$$
$$\Rightarrow |f(z_0)| =  \frac{1}{2\pi}\int_0^{2\pi} |f(z_0 + re^{it})| \,dt = \frac{1}{2\pi}\int_0^{2\pi}|f(z_0)| \,dt$$
Therefore, 
$$\int_0^{2\pi}|f(z_0)| - |f(z_0 + re^{it})| \,dt = 0$$
But 
$$ |f(z_0)| \geqslant |f(z_0 + re^{it})| $$
by our hypothesis and so our integrand is nonnegative. Therefore, the integral is zero.
$$\Rightarrow |f(z_0)| - |f(z_0 + re^{it})| = 0 \Rightarrow |f(z_0)| = |f(z_0 + re^{it})| $$
Hence, the Maximum Modulus Principle is proven. \\
\newline
\textbf{Claim.} If $f$ and $\conj{f}$ are holomorphic on $\Omega$, then $f$ is constant on $\Omega$. \\
\textbf{Proof.} By Cauchy Riemann Equations, \\
$f$ is holomorphic $\Rightarrow$ $u_x = v_y$ and $u_y = -v_x$. \\
$\conj{f}$ is holomorphic $\Rightarrow$ $u_x = -v_y$ and $u_y = v_x$. \\
This means $u_x = -u_x$, $u_y = -u_y$, $v_x = -v_x$, $v_y = -v_y$ in $\Omega$. \\
Therefore, $u$ and $v$ are constant, and so is $f(z)$. \\
\newline
\textbf{Claim.} If $|f| = C$ on $\Omega$, then $f$ is constant on $\Omega$. If $f(z_0) = 0$, then $f(z) = 0$ everywhere in $\Omega$ because $|f(z_0)| = 0$. \\
\textbf{Proof.} Let $f(z) = C$, then $g(z) = \frac{1}{f(z)}$ is holomorphic, since it is a ratio of holomorphic functions. \\
Since $|f|^2 = C^2$, then 
$$\conj{f(z)} = \frac{f(z)\conj{f(z)}}{f(z)} = \frac{|f(z)|^2}{f(z)} = C^2g(z)$$ 
is holomorphic too. Hence, $f$ is constant by the first claim. \\
Therefore, $f$ is constant along all points of distance $r$ from $z_0$. \\
\newline
\textbf{Minimum Modulus Principle.} \\
Let $f$ be analytic on a domain $\Omega \subseteq \mathbb{C}$, and assume that $f$ never vanishes. Then if there is a point $z_0 \in \Omega$ such that $|f(z_0)| \leqslant |f(z)|$ for all $z \in \Omega$, then $f$ is constant.

Let $\Omega \subseteq \mathbb{C}$ be a bounded domain, let $f$ be a continuous function on the closed set $\conj{\Omega}$ that is analytic on $\Omega$, and assume that $f$ never vanishes on $\conj{\Omega}$. Then the minimum value of $|f|$ on $\conj{\Omega}$ (which always exists) must occur on $\partial \Omega$. \\
\subsection{Mapping and $\mathbb{C}$-differentiability}
Next we want to study functions that map the open disk to itself where $f(0) = 0$. \\
\newline
\textbf{Schwarz's Lemma.}\\
If $f$ is holomorphic, $|f(z)| \leqslant 1$ on $|z| < 1$, and $f(0) = 0$, then 
\begin{equation*}
|f'(0)| \leqslant 1 \mbox{ and } |f(z)| \leqslant |z| \mbox{ on } |z| < 1
\end{equation*}
Moreover, if $|f(z_0)| = |z_0|$ for some non-zero $z_0$ or $|f'(0)| = 1$, then 
$$f(z) = e^{i\theta}z, \theta \in [0, 2\pi)$$
\textbf{Proof.} Consider
\[ 
g(z) = 
\begin{cases} 
	\frac{f(z) - f(0)}{z - 0} = \frac{f(z)}{z}, z \neq 0 \\
	f'(0), z = 0\\
   \end{cases}
\]
which we know is holomorphic on $|z| < 1$. \\
By Maximum Modulus for all $|z| < r$,
$$|g(z)| \leqslant |g(z_r)|$$
for some fixed $|z_r| = r$. 
Therefore, 
$$\frac{|f(z_r)|}{|z_r|} \leqslant \frac{1}{|z_r|} = \frac{1}{r}$$
as $r \to 1$ we have $|g(z)| \leqslant 1$ or $|f(z)$ $\leqslant |z|$ and $|f'(0)| \leqslant 1$ since $g(0) = f'(0)$. \\
If $|g(z_0)| = 1$, then $g$ has a local maximum for $|z| < 1$, 
\begin{align*}
&\Rightarrow \mbox{g is constant on } |z| < 1, |g(z)| = 1\\
&\Rightarrow |f(z)| = |z| \\
&\Rightarrow f(z) = e^{i\theta}z
\end{align*} 
\newline 
\textbf{Unit Circle Mapping (Schwarz's Lemma).} Consider $B_a(z): \{z| |z| < 1\} \to \{z| |z| < 1\}$ \\
It is a invertible, bilinear transform from the unit disk to itself, or an automorphism on the unit disk. 
$$B_a(z) = \frac{z - a}{1 - \conj{a}z}, |a| < 1$$
The function has several interesting qualities. 
\begin{align*}
|B_a(z)| &< 1 \mbox{ for } |z| < 1 \\
|B_a(z)| &= 1 \mbox{ for } |z| = 1 \\
B_a(a) &= 0\\
B_a(0) &= -a \Rightarrow B_a^{-1}(z) = B_{-a}(z) \\
B_a'(0) &= 1 - |a|^2 \\
B_a'(a) &= \frac{1}{1 - |a|^2}
\end{align*}
To use Schwarz's Lemma if $|f(z)| < 1$, but $f(z_0) = 0$, for $z \neq 0$, we try to study \\
$g(z) = f \circ B_{z_0}(z)$, $\frac{f(z)}{b_{z_0}(z)}$ or $f(z)B_{z_0}(z)$.\\
If $|f(z)| < 1$ on $\Re(z) > 0$, then consider 
$$g(z) = \frac{z + 1}{z - 1}$$
which maps $|z| < 1$ to $\Re(z) > 0$. Then we have $f \circ g^{-1}: \{z| |z| < 1\} \to \{z| |z| < 1\}$.\\
\newline
\textbf{Example.} Assume $f$ is holomorphic and bounded by 1 for $|z| < 1$, $f(\frac{1}{2}) = 0$. Estimate $|f(\frac{3}{5})|$.
Let 
\[ 
g(z) = 
\begin{cases} 
	\frac{f(z)}{B_{1/2}(z)}, z \neq \frac{1}{2} \\
	\frac{f'(\frac{1}{2})}{B_{1/2}(z)} = \frac{3}{4}f'(\frac{1}{2}), z = \frac{1}{2}\\
   \end{cases}
\]
is holomorphic on $|z| < 1$. (We used L'Hospital's Rule to define $g(\frac{1}{2})$). \\
As $|z| \to 1$, $|g| \leqslant 1$ since $|f| \leqslant 1 $ and $|B_{\frac{1}{2}}(e^{i\theta})| = 1$. \\
This means $|f(z)| \leqslant |B_{\frac{1}{2}}(z)|$ for all $|z| < 1$. \\
Hence, 
$$\abs*{f\left(\frac{3}{5}\right)} \leqslant \abs*{B_{\frac{1}{2}}\left(\frac{3}{5}\right)} = \abs*{\frac{\frac{3}{5} - \frac{1}{2}}{1 - \frac{1}{2}\cdot \frac{3}{5}}} = \frac{1}{7}$$
Note $B_{\frac{1}{2}}(z)$ is a holomorphic function with $B_{\frac{1}{2}}(\frac{1}{2}) = 0$ that hits the upper bound. Therefore, we cannot estimate better than that. \\
\newline
Now we want to use bounds on $f$ to bound $f^{(k)}$. \\
\newline
\textbf{Cauchy's Inequality.} \\
If $|f(z)| \leqslant M$ on $|z - z_0| < R$, then for $r < R$ and $C_r = \{z||z - z_0| = r\}$, 
$$\abs*{f^{(k)}(z_0)} \leqslant \frac{k!M}{R^k}$$
\textbf{Proof.}
\begin{align*}
\abs*{f^{(k)}(z_0)} &\leqslant \frac{k!}{2\pi} \oint_{C_r} \frac{|f(w)|}{|w-z_0|^{k + 1}} \, dw \\
&\leqslant \frac{k!}{2\pi}\frac{M_R}{r^{k + 1}} \oint_{C_r} \,dw\\
&=\frac{k!}{2\pi}\frac{M}{r^{k + 1}}2\pi \\
&=\frac{k!M}{r^k}
\end{align*}
Let $r \to R^-$, then 
$$\abs*{f^{(k)}(z_0)} \leqslant \frac{k!M}{R^k}$$
\newline
\textbf{Liouville's Theorem.}\\
A bounded entire function is constant. \\
\textbf{Proof.} From Cauchy's Inequality, 
$$|f'(z_0)| \leqslant \frac{M}{R}$$
then as $R \to \infty$, $f'(z_0) \to 0$. Therefore, $f$ is constant. \\
\newline
\textbf{Lemma.} Let $p(z)$ be a non-constant polynomial, then 
$$\lim_{z \to \infty} |p(z)| = \infty$$
\textbf{Proof.}
\begin{align*}
p(z) &= a_0 + a_1z + \cdots + a_nz^n\\
&= z^n(a_0z^{-n} + a_1z^{-n + 1} + \cdots + a_n)
\end{align*}
Then 
\begin{equation*}
|p(z)| = |z^n||a_0z^{-n} + a_1z^{-n + 1} + \cdots + a_n| \to \infty \mbox{ as } n \to \infty
\end{equation*}
\newline
\textbf{Proof of the Fundamental Theorem of Algebra (F.T.A).} \\
The last two results have major consequences. Let $p(z)$ be a non-constant polynomial and $p(z) \neq 0$ for all $z \in \mathbb{C}$. Then
$$f(z) = \frac{1}{p(z)}$$ 
is entire and non-constant and thus unbounded. 
As $z \to \infty$, $p(z) \to \infty$ and hence 
$$\lim_{z \to \infty} f(z) = 0$$ 
or $|f(z)| < 1$ for $|z| > R$. \\
Since $f$ is continuous on $|z| \leqslant R$, it has a bound $|f(z)| \leqslant M$, for $|z| < R$. Hence, $f(z)$ is constant and thus, $p(z)$ is constant. This is a contradiction. Therefore, $f(z)$ must not be entire. \\
Hence, the F.T.A is proven. All non-constant polynomials have a root in $\mathbb{C}$. \\
Let $p(z)$ be of degree $n$. For each root $z_i$ of $p(z)$, we can compute 
$$p(z) - p(z_i) = (z - z_i)q(z)$$
Repeat the process of $q(z)$ and show that we have exactly $n$ roots.\\
\newline
\textbf{Example.} Suppose $f$ is entire, 
$$|f'(z)| < |f(z)|$$ 
Then 
$$\frac{f'(z)}{f(z)} $$
is entire and bounded. By Liouville's Theorem, $f'(z) = kf(z)$ and $|k| < 1$. \\
Therefore, 
$$f(z) = e^{kz}, |k| > 1$$ 
\newline
\textbf{Proposition.} Suppose $f$ is entire and $|f(z)| > 1$, then $\frac{1}{f(z)}$ is entire and bounded and thus constant. \\
\newline
\textbf{Example.} Non-constant entire functions are unbounded. Therefore, $\sin(z)$ and $\cos(z)$ are unbounded since $\sin(0) \neq \sin(\frac{\pi}{2})$ and $\cos(0) \neq \cos(\frac{\pi}{2})$ \\
\newline
\textbf{Example.} Let $f$ and $g$ be entire, $\Re(h(z)) \leqslant k\Re(g(z))$ for some fixed $k$. Consider 
$$h(z) = f(z) - kg(z)$$ entire where $\Re(h(z)) < 0$. Then $e^{h(z)}$ is bounded by 1. 
$$|e^{h(z)}| \leqslant e^{\Re(h(z))} \leqslant 1$$
Therefore, $h(z)$ is constant. This means $f(z) - kg(z) = C$ for some constant $C$. 
$$f(z) = kg(z) + C$$

\newpage
\section{Convergence of Infinite Series in $\mathbb{C}$}
We want to study series in $\mathbb{C}$. 
$$\sum_{k = 0}^{\infty}a_k$$
We start with the most famous series. \\
\newline
\textbf{Geometric Series.}
$$S_n = a + ar + \cdots ar^n = \sum_{k = 0}^n ar^k$$
Observe 
\begin{align*}
(1 - r)S_n &= S_n - rS_n \\ 
&= a + ar + \cdots + ar^n - (ar + \cdots + ar^{n + 1}) \\ 
&= a(1 - r^{n + 1})
\end{align*}
Thus 
$$S_n = \sum_{k = 0}^n ar^k = \frac{a(1-r^{n + 1})}{1 - r}$$
r is the ratio of the geometric sum. \\
The geometric series is 
\begin{equation*}
S = \sum_{k = 0}^{\infty}ar^k = \frac{a}{1 - r} \mbox{ provided } |r| < 1
\end{equation*}
If $|r| \geqslant 1$, then the sum diverges and does not converge. \\
\newline
\textbf{Absolute Convergence.} \\
If $\sum {|a_n|}$ converges, then $\sum {a_n}$ converges absolutely. \\
\newline
\textbf{Example.} The following series converges absolutely. 
$$\sum \frac{(-1)^n}{n^2}$$ 
\textbf{Conditional Convergence.} \\
If $\sum {|a_n|}$ diverges and $\sum {a_n}$ converges, then the series converges conditionally. \\
\newline 
\textbf{Example.} The following series converges conditionally. 
$$ \sum \frac{(-1)^n}{n}$$
\newpage
\subsection{Convergence Tests}
\textbf{Alternating Series Test (A.S.T).} \\
Suppose that we have a series $\sum {a_n}$ and either $a_n = (-1)^nb_n$ or $a_n = (-1)^{n + 1}b_n$ where $b_n > 0$ for all n. Then if,
\begin{enumerate}[nolistsep]
\item $\lim_{n \to \infty} b_n = 0$ and,
\item$ \{b_n\}$ is a decreasing sequence
\end{enumerate}
the series $\sum a_n$ is convergent. \\
\newline
\textbf{Root Test.} \\
Suppose that we have a series $\sum {a_n}$. Define 
$$L = \lim_{n \to \infty} \sqrt[n]{|a_n|} = \lim_{n \to \infty} |a_n|^{\frac{1}{n}}$$
Then, 
\begin{enumerate}[nolistsep]
\item if $L < 1$, the series is absolutely convergent (and hence convergent). 
\item if $L > 1$, the series is divergent.  
\item if $L = 1$, the series may be divergent, conditionally convergent, or absolutely convergent. 
\end{enumerate} 
$$ \lim_{n \to \infty}n ^ {\frac{1}{n}} = 1$$ 
\textbf{Ratio Test.} \\
Suppose that we have a series $\sum {a_n}$. Define 
$$L = \lim_{n \to \infty} {\abs*{\frac{a_{n+1}}{a_n}}} $$
Then, 
\begin{enumerate}[nolistsep] 
\item if $L < 1$ the series is absolutely convergent (and hence convergent). 
\item if $L > 1$, the series is divergent.  
\item if $L = 1$, the series may be divergent, conditionally convergent, or absolutely convergent. 
\end{enumerate}
\  \\
\textbf{Basic Comparison Test (B.S.T).} \\
Suppose that we have two series $\sum {a_n}$ and $\sum {b_n}$ with $a_n, b_n \geqslant 0$ for all $n$ and $a_n < b_n$ for all $n$. Then, 
\begin{enumerate}
\item If $\sum{b_n}$ is convergent, then so is $\sum a_n$. 
\item If $\sum{a_n}$ is divergent, then so is $\sum b_n$.
\end{enumerate}
\  \\
\textbf{Limit Comparison Test (L.S.T).} \\
Suppose that we have two series $\sum {a_n}$ and $\sum {b_n}$ with $a_n \geqslant 0$, $b_n > 0$ for all $n$. Define, 
$$c = \lim_{n \to \infty} \frac{a_n}{b_n} $$
If $c$ is positive and is finite, then either both series converge or both series diverge. \\
\newline
\textbf{Integral Test.}
Suppose that $f(x)$ is a continuous, positive and decreasing function on the interval $[k, \infty)$ and that $f(n) = a_n$ then, 
\begin{enumerate}
\item If $\int_k^\infty f(x) \, dx$ is convergent so is $\sum_{n = k}^{\infty} a_n$.
\item If $\int_k^\infty f(x) \, dx$ is divergent so is $\sum_{n = k}^{\infty} a_n$.
\end{enumerate}
\newpage
\textbf{Example.} p-series. 
\begin{equation*}
\sum_{n = 1}^{\infty} \frac{1}{n^p} \mbox{ converges if and only if } p > 1.
\end{equation*}
\newline
\textbf{Monotone Convergence Theorem.} \\
If $(a_n)$ is a monotone sequence of real numbers, then $(a_n)$ is convergent if and only if $(a_n)$ is bounded. \\
Let $a_n > 0$, then $S_n = \sum_{k = 1}^n a_k$ is increasing and if it is bounded, $S_n \to S$. \\
\newline 
\textbf{Weierstrass M-Test.} \\
If $\{f_n(z)\}$ be a sequence of functions with $|f_n(z)| \leqslant M_n, z \in \Omega$ and $\sum {M_n} < \infty$, then $\sum {f_n(z)}$ converges uniformly on $\Omega$. \\
\newline
\textbf{Proposition.} \\
In general, $f_n \to f$ uniformly does imply $\int f_n \to \int f$ uniformly, but $f_n' \nrightarrow f'$, even pointwise. \\
However, for holomorphic functions $f_n \to f$ uniformly on compacta implies $f_n' \to f'$ uniformly. \\
\newline
\textbf{Example.}\\
Let 
\begin{equation*}
f_n(z) = \frac{\sin(nz)}{\sqrt{n}} \to 0 \mbox{ uniformly }
\end{equation*}
and 
$$f_n'(z) = \sqrt{n}\cos(nz), f_n'(0) \to \infty \neq f'(0)$$
\newline 
\textbf{Cauchy Condensation Test.} \\
Let $\{a_n\}$ be a series of positive terms with $a_{n + 1} \leqslant a_n$. Then $\sum_{n = 1}^{\infty} a_n$ converges if and only if 
$$\sum_{k = 0}^{\infty} 2^ka_{2^k}$$ converges. \\
\textbf{Proof.} 
\begin{align*}
&\hspace{9mm} \sum {2^nf(2^n)} < \infty \\
&\Longleftrightarrow \int_{1}^{\infty}2^x f(2^x)\,dx \\ 
&\Longleftrightarrow \log{(2)}\int_1^{\infty}2^x f(2^x) \,dx < \infty \\
&\Longleftrightarrow \int_2^{\infty} f(u) \, du < \infty \\
&\Longleftrightarrow \sum f(n) < \infty
\end{align*}
\textbf{Example.} Use Cauchy Condensation Test to prove $f(n) = \frac{1}{n}$ diverges. \\
$$f(n) = \frac{1}{n} \Rightarrow \frac{2^n}{2^n} = 1$$
\begin{equation*}
\sum_{n = 1}^{\infty} 1 = \infty \mbox{ if and only if } \sum_{n = 1}^{\infty} \frac{1}{n} = \infty
\end{equation*}
Despite $\frac{1}{n} \to 0$, the harmonic series 
$$ \sum_{n = 1}^{\infty} \frac{1}{n} = \infty$$ 
\newline
\textbf{Dirichlet Test.} \\
Let $a_1, a_2, \cdots$ be a monotonically decreasing infinite real sequence. \\
Let $\sum {b_n}$ be an infinite complex series such that its partial sums are bounded. i.e. 
$$\abs*{\sum_{n = 1}^N {b_n}} \leqslant M $$
for all $N \in \mathbb{N}$. \\
Then 
$$\sum_{n = 1}^{\infty} a_nb_n$$
converges. \\
\newline
\textbf{Example.} Determine whether the following series converge. 
$$\sum_{n = 1}^{\infty} \frac{(e^{2\pi i /3})}{\sqrt{n}}$$
When viewed as vectors, $b_1 + b_2 + b_3 = 0$ and $\sum_{n = 1}^{3k} b_n = 0$\\
We have $\abs*{\sum_{n = 1}^{\infty} b_n}$ is bounded but does not converge, just like $\abs*{\sum_{n = 1}^{\infty} (-1)^n}$. \\
Therefore, $\frac{b_n}{\sqrt{n}}$ converges. \\
Using the same technique, one can also show $\sum_{n = 1}^{\infty} \frac{\sin(n)}{n}$ converges.

\newpage
\section{Power Series in $\mathbb{C}$}
We want to express holomorphic functions in $\mathbb{C}$ by series in $z$ instead of integrals and the Cauchy Riemann Equations. Cauchy-Goursat Theorem will provide the connection between the two forms, 
$$ f(z) = e^z = \sum_{n = 0}^{\infty} \frac{z^n}{n!} $$
Let 
$$ f(z) = \sum_{n = 0}^{\infty} a_n(z - z_0)^n $$
be a power series centered at $z_0$. \\
The coefficients are unique: 
$$ a = \frac{f^{(n)} (z_0)}{n!} = \frac{1}{2\pi i} \oint_C \frac{f(w)}{(w - z_0)^{n + 1}} \, dw$$
where $C$ is a path about $z_0$. \\
\newline
\textbf{Derivative and Anit-derivative of a Power Series.} \\
$f'(z)$ and $F(z)$ are given by 
$$f'(z) = \sum_{n = 0}^{\infty} a_n \cdot n(z-z_0)^{n -1}$$
$$F(z) = \int f(z) \, dz = \sum_{n = 0}^{\infty} \frac{a_n}{n + 1}(z - z_0)^{n + 1} + C_0$$
and are holomorphic and converge for $|z -z _0| < R$. $R$ denotes the radius of convergence. \\
We have no clue about what happens with $|z - z_0| = R$, then we have to check for each $f(z)$ and point. \\
If $R = \infty$, then $f(z)$ is entire. \\
If $R = 0$, then $f$ is not holomorphic on disks centered at $z_0$. \\
\newline
Before we move on to more specifics about the radius of convergence, a brief review of terminology is in order. \\ 
A reminder about upper and lower limits. Let 
\begin{equation*}
\mbox{Largest lower bound: } \lim_{n \to \infty} \inf|a_n| = L
\end{equation*}
\begin{equation*}
\mbox{Smallest upper bound: }\lim_{n \to \infty} \sup|a_n| = U
\end{equation*}
then 
$$ L = \sup_{n \geqslant 0}(\inf_{k \geqslant n}|a_k|) = \lim_{n \to \infty} (\inf \{ |a_n|, |a_{n + 1}|, \cdots \})$$
$$ U = \inf_{n \geqslant 0}(\sup_{k \geqslant n}|a_k|) = \lim_{n \to \infty} (\sup \{ |a_n|, |a_{n + 1}|, \cdots \}) $$
and $L \leqslant U$. For $n$ large and $\epsilon_0$, $\epsilon_1 > 0$, 
$$L - \epsilon_0 \leqslant |a_n| \leqslant U + \epsilon_1$$
If we show that $U \leqslant L$, then $|a_n|$ converges.
$$\lim_{n \to \infty}|a_n| = L = U$$ 
\newline
\subsection{Radius of Convergence}
\textbf{Hadamard's Formula for $R$.} 
$$\lim_{n\to \infty} \sup|a_n|^{\frac{1}{n}} = \frac{1}{R}$$
Just like with the geometric series, 
$$ \sum_{n = 0}^{\infty} a_n(z - z_0)^n $$
converges uniformly for $|z - z_0| \leqslant r < R$ and pointwise $|z - z_0| < R$. \\
\textbf{Proof.} In this proof about convergence of power series, we could assume 
$$\lim_{n \to \infty} |a_n|^{\frac{1}{n}} = \frac{1}{R}$$
Assume $z_0 = 0$ and 
$$ L = \frac{1}{R} = \lim_{n \to \infty} \sup |a_n|^{\frac{1}{n}}$$
$L \neq 0$ or $\infty$. Let $|z| \leqslant r < R$, and from $\lim \sup$, we have 
$$|a_n|^{\frac{1}{n}} \leqslant L + \epsilon $$
for $n$ large and $\epsilon > 0$. 
Therefore, 
$$ |a_n| \leqslant (L + \epsilon)^n $$
and 
$$ |a_n||z|^n \leqslant ((L + \epsilon)|z|)^n \leqslant ((L + \epsilon)r)^n $$
Since $(L + \epsilon)r = \frac{r}{R} + \epsilon r$ and $\frac{r}{R} < 1$, we can pick $\epsilon$ small enough such that the geometric series 
$$ \sum_{n = 0}^{\infty} ((L + \epsilon)r)^n$$
converges for $|(L + \epsilon)r| < 1$. Therefore, the partial sums 
$$ |S_n(z)| = \sum_{k = 0}^n |a_kz^k|$$
are increasing sequence with respect to $n$, $z$-fixed, and bounded by a convergent sum. By the monotone convergence theorem, the series converges for $|z| < R$. By the Weierstrass M-Test, the convergence is uniform on $|z| \leqslant r < R$ but not on $|z| < R$, since we need a fixed radius. \\
\newline
\textbf{Example.} If 
$$f(z) = \sum_{n = 1}^{\infty} \frac{z^n}{n!} $$
Then $a_n = \frac{1}{n!}$
$$ \lim_{n \to \infty} \abs*{\frac{a_{n + 1}}{a_n}} = \lim_{n \to \infty} \frac{n!}{(n+1)!} = \lim_{n \to \infty} \frac{1}{n + 1} = 0 = \frac{1}{R}, R = \infty$$
Therefore, 
$$ f(z) = \sum_{n = 1}^{\infty} \frac{z^n}{n!} = e^z - 1$$
converges for all $z \in \mathbb{C}$. \\
\newpage
\textbf{Example.} If 
$$ f(z) = \sum_{n = 0}^{\infty} (-1)^n \frac{z^{2n}}{3^{2n}} $$
$$|a_n|^{\frac{1}{n}} = \frac{|z^2|}{|9|} $$
If $\frac{|z|^2}{|9|} < 1$, the sum converges absolutely on $|z| < 3$. Here, 
$$ \sum_{n = 0} ^ {\infty} (-1)^n \frac{z^{2n}}{3^{2n}} = \frac{1}{1 - (-\frac{z}{3})^2} = \frac{1}{1 + \frac{z^2}{9}} = \frac{9}{9 + z^2} = \frac{d}{dz}3\arctan(\frac{z}{3})$$
Most of the power series with reals still work here in complex. 
$$\sin(z) = \sum_{n = 0}^{\infty} \frac{(-1)^nz^{2n + 1}}{(2n + 1)!}, z \in \mathbb{C}$$
However, 
$$ \sqrt{z} \neq \sum_{n = 0}^{\infty} a_nz^n $$ 
because $\sqrt{z}$ is not differentiable at $z = 0$. \\ 
\newline
\subsection{Derivative of Power Series}
Now let us prove the claim we made before.\\ 
\textbf{Claim.} The derivative of a power series is given as below. 
$$f'(z) = \sum_{n = 0}^{\infty}na_nz^{n - 1}, |z| < R$$ 
\textbf{Proof.} Define 
$$g(z) = \sum_{n = 0}^{\infty} a_n\cdot nz^{n - 1}$$ 
and if 
$$ L = \frac{1}{R} = \lim_{n \to \infty}\sup |a_n|^{\frac{1}{n}} $$
$$ \Rightarrow \lim_{n \to \infty} \sup |na_n|^{\frac{1}{n}} = \lim_{n \to \infty}\sup |n|^{\frac{1}{n}}|a_n|^{\frac{1}{n}} = \lim_{n \to \infty}\sup|n|^{\frac{1}{n}}\lim_{n \to \infty} \sup |a_n|^{\frac{1}{n}}$$
because $|n^k|^{\frac{1}{n}} \to 1$ as $n \to \infty$, $k \in \mathbb{R}$, $k \neq 0$. Hence $g(z)$ and $f(z)$ have the same radius of convergence. 
\iffalse
$$n ^{\frac{k}{n}} = e^{\ln(n^{\frac{k}{n}})} = e^{\frac{k}{n}\ln(n)}$$
Then 
$$ \ln(n) \leqslant \sqrt{n}$$ 
$$ \ln(1) \leqslant \sqrt{1}$$ 
\begin{equation*}
\frac{1}{n} \leqslant \frac{1}{2}\sqrt{n} \mbox{ if and only if } 2\sqrt{n} \leqslant n
\end{equation*}
This implies $h(x) = \sqrt{x} - \ln(x)$ is increasing in $x$ by the Mean Value Theorem, $\ln(x) \leqslant \sqrt{x}$. \\
Therefore, 
\begin{equation*}
\frac{\ln(n)}{n} \leqslant \frac{1}{\sqrt{n}} \to 0 \mbox{ as } n\to \infty
\end{equation*}
Thus, 
$$ \lim_{n \to \infty} e^{\frac{k}{n}\ln(n)} = e^0 = 1 = \lim_{n \to \infty} n^{\frac{k}{n}} $$
by the continuity of $e^x$. \\
\fi
Now we must show $g(z)$ is the derivative of $f(z)$, 
$$f(z) = \sum_{n = 0}^N a_nz^n + \sum_{n = N + 1}^{\infty} a_nz^n = S_n(z) + E_n(z)$$ 
where $|z| < r < R$, 
$$ \frac{f(z+h) -f(z)}{h} - g(z) = \frac{S_N(z + h) - S_N(z)}{h} - S_N'(z) + (S_N'(z) - g(z)) + \frac{E_N(z + h) - E_N(z)}{h}$$
We have three terms to study, and we can set each of them to be $< \frac{\epsilon}{3}$.
\begin{equation*}
\frac{S_N(z + h) - S_N(z)}{h} - S_N'(z) \mbox{ derivative of a polynomial, } 
\end{equation*}
\begin{equation*}
S_N'(z) - g(z), S_N'(z) \to g(z) \mbox{ for } N > M \mbox{ for some } M,
\end{equation*}
\begin{equation*}
\frac{E_N(z + h) - E_N(z)}{h}, \mbox{ this is more technical and we will investigate more}
\end{equation*}
$$\abs*{\frac{E_N(z + h) - E_N(z)}{h}} \leqslant \sum_{n = N + 1}^{\infty} |a_n|\abs*{\frac{(z + h)^n - z^n}{h}} $$
This means we need a bound on $(z + h)^n - z^n$. Remember 
$$1 + x + x^2 + \cdots + x^{n - 1} = \frac{1 - x^n}{1 - x}$$
$$\Rightarrow (1 - x) (1 + x + x^2 + \cdots + x^{n - 1}) = 1 - x^n$$
If we let $x = \frac{b}{a}$, then we have a more generalized formula, 
$$a^n - b^n = (a - b)(a^{n - 1} + a^{n - 2}b + \cdots + ab^{n - 2} + b^{n - 1})$$
Apply the formula above to $a = z + h$ and $b = z$, 
$$|(z + h)^n - z^n| = |h||(z+h)^{n - 1} + (z+h)^{n - 2}z + \cdots + z^{n - 1}|$$
We have n terms of the form 
$$(z + h)^{n - 1 - k} \cdot z^k$$
and if $h$ is small, $|z+h| \leqslant r < R$, then 
$$|(z + h)^{n - 1 - k} \cdot z^k| \leqslant r^{n -1} \leqslant R^{n - 1}$$
Therefore, 
$$ |a_n|\frac{|(z+h)^n - z^n|}{|h|} \leqslant |a_n|\frac{|h|nR^{n - 1}}{|h|} = |a_n|nR^{n -1} $$
Hence, we have a bound on
$$\abs*{\frac{E_N(z+h) - E_N(h)}{h}} \leqslant \sum_{n = N + 1}^{\infty} |a_n|nr^{n - 1}$$
which we know to be the tail of a convergent sum, $g(z)$. Thus, it can be picked such that it is less than $\frac{\epsilon}{3}$ because the tail tends to 0 as $n \to \infty$. Hence, 
$$\abs*{\frac{f(z+h)-f(z)}{h} - g(z)} < \epsilon$$
or $f'(z) = g(z)$. \\
\newline
\textbf{Proposition.} \\
Power series functions are holomorphic in $|z - z_0| < R$ and are naturally $\infty$-differentiable, i.e. $\mathbb{C}^\infty$.

\newpage
\section{Series Expansion of Holomorphic Functions}
Now we want to show if $f(z)$ is holomorphic on $\Omega$, it has a power series expansion on any disk in $\Omega$. \\
\newline
\textbf{Connection of the Geometric Series to Cauchy's Integral Formula.} \\
$$f(z) = \frac{1}{2\pi i}\oint_C \frac{f(w)}{w - z} \,dw$$
$$\frac{1}{w - z} = \frac{1}{w} \cdot \frac{1}{1 - \frac{z}{w}} = \frac{1}{w} \left(1 + \frac{z}{w} + \left(\frac{z}{w}\right)^2 + \cdots \right), r = \frac{z}{w}$$
If $|\frac{z}{w}| < 1$, $|z| < |w|$, the sum converges uniformly, 
\begin{align*}
\frac{1}{w - z} &= \frac{1}{w}\sum_{n = 0}^{\infty}\left(\frac{z}{w}\right)^n \\
\Rightarrow f(z) &= \frac{1}{2\pi i}\oint_C \frac{f(w)}{w - z} \,dw = \frac{1}{2\pi i} \oint_C \frac{1}{w} \sum_{n = 0}^{\infty} f(w) \left(\frac{z}{w} \right)^n \,dw \\
&= \frac{1}{2\pi i}\sum_{n = 0}^{\infty} \oint_C f(w)\frac{1}{w}\left(\frac{z}{w}\right)^n \,dw \mbox{ because of uniform convergence} \\
&= \sum_{n = 0}^{\infty} \frac{1}{2\pi i}\oint_C \frac{f(w)}{w^{n + 1}}\,dw z^n \\
&= \sum_{n = 0}^{\infty}a_nz^n
\end{align*}
where 
$$a_n = \frac{1}{2\pi i}\oint_C\frac{f(w)}{w^{n + 1}}\, dw = \frac{f^{(n)}(0)}{n!}$$
\newline
\textbf{Theorem.} \\
If $f$ is holomorphic on $\Omega$, then for every disk in $\Omega$ centered at $z_0$, $f$ has a unique power series expansion at $z_0$ that converges to $f(z)$. \\
If the disk is closed and in $\Omega$, then the power series converges uniformly to $f(z)$. \\
\newline
\textbf{Isolated and Non-isolated Root. } \\
Let $f(z_0) = 0$. If $f$ does not have a root on $0 < |z - z_0| < \delta$ for some $\delta > 0$, then the zero is \textit{isolated}. This means $f$ does not have a root on punctured disk centered at $z_0$. \\
If $f$ has a root on $0 < |z - z_0| < \delta$ for all $\delta > 0$, then the zero is \textit{non-isolated}. $f$ has a root on every punctured disk centered at $z_0$. We can say that $f$ has a limit point of zeros at $z_0$. \\
\newline
\textbf{Example.} $f(z) = z(z-3) $ has isolated roots at $z = 0, 3$. \\
\newline
\textbf{Theorem.} \\
Suppose $f(z)$ is holomorphic on $\Omega$ and there exists a sequence $\{z_n\}$ in $\Omega$ where $z_n \to z_0 \in \Omega$ and $f(z_n) = 0$ for all $n$. Then $f(z) = 0$ on $\Omega$. \\
\textbf{Proof.} Let 
$$f(z) = \sum_{n = 0}^{\infty} a_n(z-z_0)^n$$
Suppose $f$ is not the zero function in the disk of convergence. That means $a_n \neq 0$ for some $n \in \mathbb{N}$. Let $m \in \mathbb{N}$ be the smallest number such that $a_m \neq 0$. In other words, $f(z)$ has a root of order $m$ at $z_0$. Then, 
$$f(z) = a_m(z - z_0)^m + \sum_{n = m + 1}^{\infty} a_n(z - z_0)^n = a_m(z - z_0)^m\left(1+\sum_{n = m + 1}^{\infty}\frac{a_n}{a_m}\frac{(z - z_0)^n}{(z - z_0)^m}\right) = a_m(z-z_0)^m(1+g(z))$$
by letting
$$g(z) = \sum_{n = m + 1}^{\infty}\frac{a_n}{a_m}\frac{(z - z_0)^n}{(z - z_0)^m}$$
so $g(z) \to 0$ as $z \to z_0$. The function $g(z)$ exists and is holomorphic and continuous since $m < n$ in the sum and $a_m \neq 0$. \\
Consider our sequence $z_k$ that converges to $z_0$, then 
$$a_m(z_k - z_0)^m \neq 0$$ 
since $z_k \neq z_0$ is clearly not a root of $(z - z_0)^k$ and 
$$1 + g(z_k) \to 1 + 0 = 1$$
because $g(z)$ is continuous and is tending to zero. In particular, we can find a neighborhood of $z_0$ containing $z_k$ for $k > N$ such that 
$$|g(z_k)| < \frac{1}{2} \Rightarrow |1 + g(z_k)| > \frac{1}{2} \Rightarrow 1 + g(z_k) \neq 0$$ 
Therefore, $f(z_k) \neq 0$ and $f(z_k) = 0$ for all $k > N$. This is a contradiction. \\
Hence, it is proven that $a_n = 0$ for all $n \in \mathbb{N}$. \\
\newline
\iffalse
Locally, $f(z) = 0$ near the limit point of zeros. Let $U$ be the interior of the points where $f(z) = 0$, $U$ is open and non-empty from what is showed above. The power series expansion is for the zero function and the expansion is valid on an open disk centered at $z_0$. 
\fi
\textbf{Uniqueness Theorem.} \\
Let $f$ and $g$ be holomorphic on $\Omega$. The following are equivalent: 
\begin{enumerate}
\item $f(z) = g(z)$ for all $z \in \Omega$.  
\item $f(z) = g(z)$ for all $z \in U$, where $U$ is a non-empty, open subset of $\Omega$. 
\item For some $a \in \Omega$, $f^{n}(a) = g^{n}(a)$ for all $n \in \mathbb{N}$. 
\item $f(z_n) = g(z_n)$ for some $z_n \to z_0$ in $\Omega$, with $z_n$ distinct. 
\end{enumerate}



\newpage
\section{Open Mapping Theorem and Reflection Principle}
\section{Laurent Series}
\section{Residue Theorem}
\section{Improper Integrals}
\section{Argument Principle and Rouche's Theorem}

\iffalse
Introduction to Myself
\subsection{Subsection}
Education Background
\subsubsection{Subsubsection} 
More Text
\paragraph{Paragraph}
Some more text 
\subparagraph{Subparagraph} 
Even more text 
\section{Another Section} 
\fi

\begin{equation*}
	f(x) = x^2
\end{equation*}
this formula is an example $f(x) = x$
\begin{align*}
1 + 2 &= 3\\
1 &= 3 - 2
\end{align*}
\begin{align*}
f(x) &= x ^ 2\\
g(x) &= \frac{1}{x}\\
h(x) &= \int^a_b \frac{1}{x}x^3\\
F(x) &= \frac{1}{\sqrt{x}}
\end{align*}

\begin{align*}
\left[
\begin{matrix}
1 & 0\\
0 & 1
\end{matrix}
\right] \\
\left(\frac{1}{\sqrt{x}}\right)
\end{align*}

\newpage
Core Material:
1. Finding patterns in data; using them to make predictions. 
2. Models and statistics help us understand patterns. 
3. Optimization algorithms "learn" the patterns.

Classification: 
1. 

\end{document}